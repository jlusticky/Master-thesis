%=========================================================================
% (c) 2014, 2015 Josef Lusticky

\chapter{Conclusion}\label{chap:conclusion}
The thesis provides description of the principles behind packet processing in the Linux kernel, as well as
a general description of the 40 Gigabit Ethernet protocol and its performance limitation.
A detailed knowledge of the principles described in the text referenced above is required to
perform necessary system settings for achieving maximum routing performance using the GNU/Linux operating system.


The overall routing performance of the Linux kernel is sufficient for routing %
Gbps of iMix traffic using the hardware equipment described in section~\ref{sec:analysis-hardware}.


40Gbit software routing using GNU/Linux performs. %provides a reasonable alternative to expensive hardware routers.
The main bottleneck of the default installation is the {\it{irqbalancer}} daemon, which. %TODO
The thesis can be further extended through comparision of the GNU/Linux routing solution
against hardware-based routers.
These measurements could focus on comparing latency difference, throughput or power consumption.

There are various user-space frameworks focused on improving the GNU/Linux network performance
targeted for particular tasks, such as Data Plane Development Kit~\cite{dpdk}.
Most of these frameworks are focused on improving the throughput and latency of user-space application,
however, various extensions to perform a framework based routing exists.
Further work extending this thesis could include comparing these frameworks with the results presented in this thesis.

Another further work can be focused on replacing the Spirent hardware packet generator with
a software solution.
There are various high-speed packet generators avaliable for GNU/Linux
such as well-known pkt-gen, which is a part of the upstream kernel,
or netmap, which claims that is can generate 14.88 Mpps using a single CPU's core~\cite{netmap}.
