%=========================================================================
% (c) 2014, 2015 Josef Lusticky

\chapter{Conclusion}\label{chap:conclusion}
The thesis provides description of the principles behind packet processing in the Linux kernel, as well as
a general description of the 40 Gigabit Ethernet protocol and its performance limitation.
An advanced knowledge of the principles described in the thesis is required to
perform the correct system settings for maximum routing performance with the GNU/Linux operating system.

The CentOS~7 operating system was installed to perform the measurements.
The system features Linux kernel based on version 3.10.
Additionally, upstream kernel 4.0.2 was installed.
The Linux kernel routing performance was measured under different scenarios, including
simulation of a real internet traffic, based on the data provided by the Amsterdam Internet Exchange.
A custom frame-size distribution was configured for this purpose and used in the experiments.

40Gbit software router with GNU/Linux may provide a reasonable alternative to proprietary hardware-based routers.
There are several settings of the default installation which negatively impact routing performance.
The thesis describes how to adjust the settings for the purpose of maximum throughput.

The overall routing performance of the Linux kernel is sufficient for routing 35.91~Gbps
of the simulated Internet traffic on Intel Xeon CPU, that is 5.9~million frames per second.
The CPU features 8~physical cores and the Hyper-Threading technology (16~ logical cores).
The Mellanox ConnectX-3 EN adapter is able to scale the packet processing up to 16~cores,
however, the measurements show that utilising 8~cores performs slightly better.
Since the packet processing is highly optimised in terms of cache hit ratio,
this may be a constraint of the Hyper-Threading technology.
The measurement involving the Internet BGP routes shows,
that the Linux kernel is sufficient for routing 30.95~Gbps of the simulated Internet traffic,
that is 5.1~millions frames per second.

The Linux kernel uses advantage of its scaling mechanisms to perform well in network processing,
however, a single-core packet processing must be improved to achieve better results.
The Linux kernel is able to route approx.~1~150~000
frames per second utilising a single physical core.
This is also the limitation of a single network flow processing,
since the scaling mechanisms implemented in the Linux kernel and network adapters
are based on processing each flow by a different CPU.

The thesis can be further extended through a comparison against hardware-based routers.
The future work may compare latency, throughput or power consumption of both systems.
Additionally, various user-space frameworks focused on improving the GNU/Linux network performance may be evaluated,
such as Data Plane Development Kit.
Further work extending the thesis may include comparison of
the frameworks with the presented results.
