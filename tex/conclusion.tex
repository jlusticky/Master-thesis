%=========================================================================
% (c) 2014, 2015 Josef Lusticky

\chapter{Conclusion}\label{chap:conclusion}
The thesis provides description of the principles behind packet processing in the Linux kernel, as well as
a general description of the 40 Gigabit Ethernet protocol and its performance limitation.
An advanced knowledge of the principles described in the thesis is required to
perform the correct system settings for maximum routing performance using the GNU/Linux operating system.

The CentOS~7 operating system was installed to perform the measurements.
The system features Linux kernel based on version 3.10.
Additionally, upstream kernel 3.19.2 was installed.
The Linux kernel routing perfomance was measured under different scenarios, including
simulation of a real internet traffic, based on the data provided by the Amsterdam Internet Exchange.
Custom frame-size distribution was configured for this purpose and used in the experiments.
%TODO BGP
The overall routing performance of the Linux kernel is sufficient for routing %TODO
Gbps of the configured traffic using Intel Xeon CPU with 16 logical cores (8~physical cores and HyperThreading enabled).
That is, the Linux kernel is able to route approx.~5~millions frames per second using 16 cores.

40Gbit software routing using GNU/Linux may provide a reasonable alternative to proprietary hardware-based routers.
The main bottleneck of the default installation is the {\it{irqbalancer}} daemon, which
does not set the optimal IRQ affinity for networking purposes.
Thus, the scaling mechanisms implemented in the Linux kernel are not used optimally.
The thesis describes how to set the IRQ affinity manually for the purpose of maximum throughput.

The Linux kernel uses advantage of its scaling mechanisms to perform well in network processing,
however, a single-core packet processing can be improved to achieve better results.
The Linux kernel is able to route approx.~500~000 %TODO
frames per second utilising a single core.
This is also the limitation of a single network flow processing,
since the scaling mechanisms implemented in the Linux kernel are based on processing each flow by a different CPU.

The thesis can be further extended through comparision against hardware-based routers.
The future work may compare latency, throughput or power consumption of both systems.
Additionally, various user-space frameworks focused on improving the GNU/Linux network performance may be evaluated,
such as Data Plane Development Kit.
Further work extending this thesis may include comparing the performance of
the frameworks with the results presented in the thesis.

%Another further work can be focused on replacing the Spirent hardware packet generator with
%a software solution.
%There are various high-speed packet generators avaliable for GNU/Linux
%such as well-known pkt-gen, which is a part of the upstream kernel,
%or netmap, which claims that is can generate 14.88 Mpps using a single CPU's core~\cite{netmap}.
