%=========================================================================
% (c) 2014, 2015 Josef Lusticky

\section{Compatibility}\label{sec:40gbe-compatibility}
The previous 10~Gigabit Ethernet standard was ratified by the IEEE 802.3 Working Group in 2002~\cite{ieee-802.3ae}.
As opposed to 40~GbE, the cabling plant of 10~GbE uses just a single pair of fiber.
In 2013, Cisco introduced a replacement that does not require a change to the cabling plant
and makes it possible to run 40~GbE over a single pair of multimode fiber~\cite{40gbe-mmf}.

Since its ratification in 2002,
it took four years to standardise 10~GbE over copper twisted pair 10GBASE-T in 2006~\cite{ieee-802.3an}.
10GBASE-T can run over a Category~6 cable within the range of 55m.
For full 100m range, a Category~6a cable is required~\cite{ieee-802.3an}.
One of the previous barriers to 10GBASE-T adoption was the power consumption per port compared to other 10 GbE options.
However, improvements in semiconductor manufacturing technology
significantly decreased power use to the point where it is no longer a concern.
Nowadays, 10GBASE-T and Category~6A cabling costs less than optical fiber technology~\cite{belden-10g-40g}.

The 40~GbE over Category~8 copper twisted pair
is currently being discussed by the IEEE P802.3bq 40GBASE-T Task Force~\cite{ieee-802.3bq}.
The maximum range for 40GBASE-T is planned to be just 30 meters.
Such range should be sufficient for most switch-to-server connections in data centres~\cite{ieee-802.3bq}.
Because of the shorter distance, the power consumption of 40GBASE-T should be less than of 10GBASE-T~\cite{belden-10g-40g}.

40~Gigabit Ethernet is backward compatible with its predecessor.
Due to concerns around vendor and equipment interoperability,
IEEE has determined they will not support or define Jumbo frames~\cite{ea-jumbo-frames}.
Modern 100Mbps or higher Ethernet also uses constant signalling, which avoids the need for the preamble~\cite{anatomy-frame}.
However, the frame format is preserved for today's Ethernet transmission speeds to avoid making any changes.
With the upcoming 40GBASE-T variant, 10M/100M/1G/10G/40G link speed auto-negotiation can be expected in the future.

The 40~Gigabit Ethernet protocol support was introduced in the Linux kernel by various NIC vendors.
Red Hat Enterprise Linux 7 supports 40 Gigabit network interface controllers since its initial release~\cite{rhel-7-announce}.
Additional support was also introduced in RHEL 6.6~\cite{rhel-66-announce}.
