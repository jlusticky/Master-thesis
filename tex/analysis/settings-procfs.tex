%=========================================================================
% (c) 2014, 2015 Josef Lusticky

\subsection{Procfs settings}\label{sub:analysis-settings-procfs}
The variables exported via procfs are accessible as files under /proc in CentOS~7.
The variables in the /proc/sys directory can be manipulated by the {\it{sysctl}} utility as well.
The files in the /proc/sys/net directory are of interest for the experiments.
This directory includes the following subdirectories:
\begin{itemize}
\item /proc/sys/net/ipv4 - contains variables influencing the IPv4 protocol settings
\item /proc/sys/net/ipv6 - contains variables influencing the IPv6 protocol settings
\item /proc/sys/net/netfilter - contains variables influencing the netfilter settings, not discussed in this thesis
\item /proc/sys/net/unix - contains variables influencing communication over unix sockets, not discussed in this thesis
\item /proc/sys/net/core - contains variables influencing low-level networking settings, including parameters of NAPI, Low Latency Sockets, etc.
\end{itemize}

The most important setting for the routing performance measurements of the Linux kernel is IPv4 forwarding.
It can be enabled by writing "1" to the /proc/sys/net/ipv4/ip\_forward file.
This variable is special - its change resets all IPv4 configuration parameters to their default state~\cite{kernel-doc-ip-sysctl}.
The /proc/sys/net/ipv4/conf/{\it{ifname}}/forwarding file can be used
to further selectively enable or disable forwarding on a particular interface.
Historically, some of the files in /proc/sys/net/ipv4 also influence settings of L4 protocols,
such as memory limits, TCP Timestamping, Selective ACKs, etc.
Although these files are located in the {\it{ipv4}} subdirectory, the L4 settings are independent on the underlying protocol.
Most of the Layer~4 settings are auto-tuned by the kernel itself and
their description is outside the scope of this thesis~\cite{linux-kernel-networking}.

Files in the {\it{ipv6}} directory influence the IPv6 protocol settings only.
The IPv6 protocol is disabled in CentOS 7 on all interfaces by default.
To enable the IPv6 protocol on all interfaces,
the variable accessible via /proc/sys/net/ipv6/conf/all/disable\_ipv6 must be changed to "0".
Similarly, the /proc/sys/net/ipv6/conf/all/forwarding variable must changed to "1" to enable IPv6 forwarding on all interfaces.
Both settings can be changed on per-interface basis as well.

The /proc/sys/net/ipv4/route.max\_size sets the maximum number of IPv4 routes allowed in the kernel.
This is 2~147~483~647 by default in CentOS~7, which is enough for a full BGP table,
which contains approx.~538~000 prefixes at the time of writing~\cite{bgp-analysis-reports}.
The /proc/sys/net/ipv6/route.max\_size sets the maximum number of IPv6 routes allowed in the kernel.
This is 4096 by default in CentOS~7, which must be raised for the measurements involving IPv6 BGP routes.
The number of IPv6 prefixes announced in the Internet is approx.~22~000 at the time of writing~\cite{bgp-analysis-reports}.

The source IPv4 address validation is enabled by default in CentOS 7.
This feature is called Reverse path filtering ({\it{rp\_filter}}) in the Linux kernel and it prevents IP spoofing.
However, it introduces additional processing and thus it should be disabled during the experiments.
The {\it{rp\_filter}} can be disabled on a particular interface
by writing "0" to /proc/sys/net/ipv4/conf/{\it{ifname}}/rp\_filter~\cite{kernel-doc-ip-sysctl}.
The {\it{rp\_filter}} for IPv6 is implemented in the netfilter subsystem of the Linux kernel and
thus it can be configured by ip6tables~\cite{kernel-source}.

Files in core directory provide access to low-level variables of the networking code.
There are two parameters that influence NAPI processing.
The /proc/sys/net/core/dev\_weight file sets the maximum number of packets that a single device
can feed to the kernel in its {\it{poll()}} function.
The default value is 64 in CentOS 7.
This value can be increased to allow the device to feed more packets at once.
However, most of the drivers provide their own limit which cannot be overwritten unless the code of the driver is changed.
This is the case of the mlx4 driver as well, which imposes the limit to 64 packets~\cite{kernel-source}.
The /proc/sys/net/core/netdev\_budget file sets the
maximum number of packets taken from all interfaces by a single {\it{net\_rx\_action()}} run.
The interfaces which are registered to polling are
probed in a round-robin manner, as described in subsection~\ref{sub:linux-ingress-napi}.
To allow the kernel to spend more time on packet processing, the {\it{netdev\_budget}} value can be increased.

The proc filesystem further provides access to the IRQ settings and statistics.
The /proc/interrupts file exports a table of all registered interrupts and their respective counters for each CPU.
Each registered interrupt has its own IRQ number.
On a multiprocessor system the interrupt can be served by any of the present CPU if the physical bus supports it.
This is the case of the PCI-Express MSI-X feature, which allows to deliver an interrupt to a specified CPU,
as described in section~\ref{sec:40gbe-throughput}.
The /proc/interrupts file even allows to assign IRQ to a CPU which is not directly connected to
the PCI-Express, however this would lead to additional overhead due to inter-processor interrupts and
QPI communication. Routing performance of such configuration can be measured.

Section~\ref{sec:linux-scaling} described how a network adapter with multiple queues
allows to scale the network traffic processing on a multi-processor system.
Each queue has assigned its own separate IRQ, thus it can be served by a particular CPU if configured properly.
The targeted CPU is defined using a mask written to the /proc/irq/{\it{NUMBER}}/smp\_affinity file,
or using a list of CPUs written to the /proc/irq/{\it{NUMBER}}/smp\_affinity\_list file.
The /proc/irq/default\_smp\_affinity file specifies a
default mask of CPUs for newly registered interrupts~\cite{kernel-doc-irq-affinity}.

The Linux kernel does not set the IRQ mapping automatically.
Instead, the user must configure it manually.
The {\it{irqbalance}} daemon reads the content of the /proc/interrupts file
and assigns the IRQ mappings according to the load.
The {\it{irqbalance}} daemon is part of the CentOS~7 and it is enabled by default.
While the {\it{irqbalance}} can introduce some performance advantage,
it does not take scaling or CPU placements (NUMA) into account~\cite{irqbalance-source}.

The CentOS~7 operating system further features the {\it{tuned}} utility in its default installation.
The {\it{tuned}} allows user to switch between user definable tuning profiles.
Several predefined profiles are already included, such as {\it{network-throughput}} or {\it{network-latency}}.
However, none of the profiles influences the low-level packet processing parameters described in chapter~\ref{chap:linux}.
Instead, the {\it{tuned}} focuses on L4 protocol parameters such as socket memory options~\cite{tuned}.
