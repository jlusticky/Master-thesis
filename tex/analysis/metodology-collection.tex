%=========================================================================
% (c) 2014, 2015 Josef Lusticky

\subsection{Statistics collection}
The Spirent TestCenter Application is able to display counters of transmitted and received frames on each interface.
The counters can be used to determine whether all of the transmitted packets on one interface were received
on the other interface and hence successfully forwarded by the server.
Unfortunately, the provided Spirent TestCenter Application contains no licence to perform the RFC 2544 throughput test
automatically, so the measurements must be configured manually in the Spirent TestCenter application.
The manual configuration consists of configuring the transmit rate, observing the packet counters and comparing their values.
If the server forwards packets without a single loss, the transmit rate can be increased and the test repeated.
%Otherwise, the transmit rate must be decreased.

The proc filesystem provides access to several statistics as well.
The network statistics exported via proc filesystem can be found in the /proc/net directory.
The files in this directory are read-only and cannot be manipulated using the sysctl utility.
There are 2 important files for the measurements - the {\it{fib\_trie}} and {\it{fib\_triestat}}.
The {\it{fib\_trie}} file exports the kernel's Forwarding Information Base overview.
The file describes both the Main and Local FIBs, as described in section~\ref{sec:linux-routing}.
The {\it{fib\_triestat}} file exports metadata of the FIB trie structures,
such as average depth, maximum depth, number of leaves, etc~\cite{kernel-source}.

The CPU utilisation can be observed using the perf utility,
which can be found in the tools/perf directory of the Linux kernel source code.
Although perf is included in the Linux kernel source code,
it must be installed separately on most GNU/Linux distributions, including CentOS~7.
To obtain additional statistics, such as PCI-Express utilisation or memory utilisation,
the Intel Performance Counter Monitor (PCM)\footnote{\url{https://software.intel.com/en-us/articles/intel-performance-counter-monitor-a-better-way-to-measure-cpu-utilization}}
can be used.
The PCM package is not included in the official CentOS~7 repository,
but it can be compiled directly from the source code.
Non-maskable interrupt watchdog must be disabled
in order to run Intel PCM.
The non-maskable interrupt watchdog can be disabled by writing "0" to the /proc/sys/kernel/nmi\_watchdog file.
