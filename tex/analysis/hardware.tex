%=========================================================================
% (c) 2014, 2015 Josef Lusticky

\section{Hardware equipment}\label{sec:analysis-hardware}
The network interface card used in the experiments is
Mellanox ConnectX-3 EN QSPF dual-port PCI-E 3.0 x8 MCX314A-BCBT~\cite{mellanox-product-brief}.
The card was provided by the Faculty of Information Technology, Brno University of Technology.
Mellanox ConnectX-3 EN is an adapter that can run 10~Gigabit Ethernet and 40~Gigabit Ethernet.
It also supports nonstandard 56~Gbps link speed while connected to Mellanox switches.
The card is PCI-Express 3.0 x8 compatible with support of previous PCI-Express versions.
Mellanox ConnectX-3 EN is a multiqueue NIC with MSI-X support %TODO up to 16 receive queues per port
and it features Receive Side Scaling with hashing support for both IPv4/IPv6 and TCP/UDP flows~\cite{mellanox-user-manual}.

The Mellanox NIC requires PCI-Express 3.0 x8 slot to take full advantage of its speed.
Brno University of Technology provided a server with
the Supermicro X10DRU-i+ motherboard, which
features PCI-Express 3.0 slots compatible with the Mellanox Connect-X 3 EN adapter~\cite{supermicro-board}.
The server is further equipped with two Intel Xeon E5-2660 v3 processors at 2200MHz with 10 physical cores per CPU
and 20 logical cores per CPU when Hyper-Threading is enabled.
Each CPU has 20MB shared L3 cache and PCI Express 3.0 support with up to 40 lanes~\cite{intel-xeon-cpu}.

There are various software frameworks for high-speed packet generation.
Kernel-space frameworks are expected to outperform user-space tools, which would always introduce some necessary delays.
The pktgen project is an upstream component of the Linux kernel.
At the time of writing, Pktgen itself is not capable of generating even full 10~GbE frame rates~\cite{netmap}.
However, it can be combined with other frameworks for fast packet processing, such as
Intel's Data Plane Development Kit~\cite{dpdk}.

Another kernel-space packet generator is Netmap,
which is not an upstream part but provides patches to the Linux kernel~\cite{netmap}.
Netmap claims to generate 14.88~million frames per second, which is a full frame rate of 10~Gigabit Ethernet,
however it was not tested against 40~GbE full frame rate of 59.5~million frames per second,
as calculated in section~\ref{sec:40gbe-frame-rates}.
Although both above mentioned frameworks seem promising, their benchmarking and description
are outside the scope of this thesis.
Moreover, to perform the measurements using a software-based packet generation,
another GNU/Linux server and a 40~GbE NIC is needed.

Another solution is to use a hardware-based packet generator such as Spirent~\cite{spirent}.
With kind permission of CESNET, the Czech national research and education network operator,
the Spirent SPT-3U equipped with a combined 100Gb / 2x40Gb Ethernet module was used to perform the measurements.
The Spirent packet generator supports generation of custom Layer 2-7 traffic, custom frame length and various
predefined traffic patterns with variable frame length called Internet Mix (iMix).
These patterns represent a typical distribution of frame lengths found in the Internet traffic
and they can be further customised.
Spirent SPT-3U further supports custom frame rate definition and bandwidth use specification~\cite{spirent}.
