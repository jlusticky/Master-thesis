%=========================================================================
% (c) 2014, 2015 Josef Lusticky

\section{Software equipment}\label{sec:analysis-software}
The Red Hat Enterprise Linux 7 operating system features 40~Gb Ethernet support and
the kernel based on upstream version 3.10~\cite{rhel-7-announce}.
To avoid licensing fees, CentOS 7 can be installed.
CentOS 7 provides support for the 40 Gigabit Ethernet protocol and a binary compatible kernel with RHEL 7~\cite{centos-7-announce}.
Since the kernel version 3.10 was released in 2013, the latest upstream kernel can be installed
to provide additional comparison.
The latest upstream kernel can be either compiled directly from the source code or downloaded from a third party repository.
The ELRepo repository contains the {\it{kernel-ml}} package
with the latest upstream kernel version, which is 4.0.2 at the time of writing~\cite{elrepo-kernel-ml}.

Mellanox ConnectX-3 EN is supported by the mlx4 driver found in the
{\it{drivers/net/ethernet/mellanox/mlx4}} directory of the Linux kernel source code.
It is the low level driver implementation for the Connect-X adapters designed by Mellanox Technologies.
Some Connect-X adapters can operate as an InfiniBand adapter and as an Ethernet NIC.
To accommodate the two flavors, the driver is split into modules mlx4\_core, mlx4\_en and mlx4\_ib.
The mlx4\_core module handles low-level functions like device initialisation and firmware commands processing.
The mlx4\_en module handles Ethernet specific functions and
plugs into the network device layer of the Linux kernel.
Similarly, the mlx4\_ib module handles InfiniBand specific functions~\cite{mellanox-user-manual}.
By default, the driver uses adaptive interrupt moderation for the receive path,
which adjusts the interrupt moderation to the traffic pattern~\cite{mellanox-user-manual}.

Spirent TestCenter Application version 4.46 is provided to use the Spirent hardware packet generator.
The application allows to create virtual devices connected to selected ports.
The devices can use a full-featured IPv4 or IPv6 stack, including ARP/ND, ICMP, TCP/UDP etc~\cite{spirent-app}.
