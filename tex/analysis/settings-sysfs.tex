%=========================================================================
% (c) 2014, 2015 Josef Lusticky

\subsection{Sysfs settings}\label{sub:analysis-settings-sysfs}
Apart from the {\it{proc}} filesystem, the Linux kernel provides another virtual filesystem found under /sys in CentOS~7.
The {\it{sys}} filesystem exports information about loaded modules, including the parameters if a module takes any.
The /sys/modules/mxl4\_core/parameters directory contains parameters used by the mlx4\_core module.
The {\it{msi\_x}} parameter is set to 1 by default, which means attempt to use MSI-X.
The /sys/modules/mxl4\_en/parameters directory contains parameters used by the mlx4\_en module.
The {\it{udp\_rss}} parameter is set to 1 by default, which enables RSS for incoming UDP traffic.
There are more parameters taken by both modules, but they are not discussed in this thesis.
Further description of the parameters can be found
in the {\it{drivers/net/ethernet/mellanox/mlx4}} directory of the Linux kernel source code~\cite{kernel-source}.

Each network interface is represented by a symlink in the /sys/class/net/ directory.
The symlinks point to the corresponding network device, which is represented as a directory in {\it{sysfs}}.
The scaling mechanisms described in section~\ref{sec:linux-scaling} can be set
using the files exported in /sys/class/net/{\it{ifname}}/queues.
Each rx-{\it{xx}} subdirectory represents a single hardware receive queue.
The rx-{\it{xx}}/rps\_cpus file can be used to set a mask of CPUs serving interrupt requests from a particular hardware queue.
Since the Mellanox ConnectX-3 NIC supports RSS, the RPS feature is disabled by default - each rps\_cpus is set to 0.
Similary, the XPS feature can be configured via tx-{\it{xx}}/xps\_cpus.
The XPS configuration should be always checked to reflect the IRQ affinity mappings configured via proc.
