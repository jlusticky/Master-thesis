%=========================================================================
% (c) 2014, 2015 Josef Lusticky

\section{Software settings}\label{sec:analysis-settings}
The CentOS 7 operating system features various components that influence forwarding performance.
The performance of the CentOS~7 distribution kernel is measured in this thesis.
Further measurements also take the latest upstream kernel 3.19.2 into account~\cite{kernel-source}.

To measure a bare routing performance of the Linux kernel,
the Netfilter and SELinux components should be disabled.
If disabling the Netfilter is not appropriate,
the {\it{iptables}} utility must be used to configure the Netfilter to allow IP forwarding,
because the default rules do not allow packet forwarding in CentOS~7.
Additionally, SELinux should be disabled to prevent performance decrease.
SELinux in CentOS~7 uses enforcing policy by default.
An influence of both the components on forwarding performance can be measured.

The Linux kernel features dynamic CPU frequency scaling.
The CPUFreq governors are policies that decide what frequency should be used.
The CPUfreq governor {\it{performance}} should be used during the measurements.
The {\it{performance}} governor sets the CPU statically to the
highest frequency available. %, which is 2.6~GHz in case of Intel Xeon E5-2630 v2.
The default governor is {\it{powersave}} in CentOS~7,
which sets the CPU statically to the lowest frequency~\cite{cpufreq-governors}.

To change the Linux kernel compile-time configuration, the kernel must be recompiled.
The CentOS~7 kernel provides a fair amount of features that could break existing setups when disabled.
The default kernel compile-time configuration does not have to provide the best routing performance,
however, it is usually used in most scenarios and hence its benchmark results are of interests for most people.

%=========================================================================
% (c) 2014, 2015 Josef Lusticky

\subsection{CentOS~7 kernel compile-time configuration}
When the Linux kernel is compiled with support for symmetric multiprocessing
with the CONFIG\_SMP option and runs on a multiprocessor system, the code for receiving and
transmitting packets takes full advantage of that.
Every modern GNU/Linux distribution compiled for the AMD64 architecture has the option enabled,
including the CentOS~7~\cite{understanding-internals}.
Advanced networking features enabled in compile-time
does not have to be used in run-time.
For example, the MULTIPLE\_IP\_TABLES support is enabled in the CentOS 7 distribution kernel, however,
since the measurements presented in this thesis use no policy routing,
the simple FIB lookup principle described in section~\ref{sec:linux-routing} is still performed.

The CentOS~7 Linux kernel configuration has the CONFIG\_NO\_HZ\_FULL option enabled.
That is, the system uses adaptive ticks
and there are no regular interrupts from the timer
which would cause additonal delays during the packet processing~\cite{kernel-doc-nohz}.

Apart from compile-time options, the Linux kernel configuration can be changed in run-time.
The {\it{proc}} and {\it{sys}} filesystems provide access to the kernel variables that influence packet processing.
Tuning of these variables can provide a significant performance improvement when configured properly.


%=========================================================================
% (c) 2014, 2015 Josef Lusticky

\subsection{Procfs settings}\label{sub:analysis-settings-procfs}
The variables exported via procfs are accessible as files under /proc in CentOS~7.
The variables in the /proc/sys directory can be manipulated by the sysctl utility as well.
The files in the /proc/sys/net directory are of interest for the experiments.
This directory includes the following subdirectories:
\begin{itemize}
\item /proc/sys/net/ipv4 - contains variables influencing the IPv4 protocol settings
\item /proc/sys/net/ipv6 - contains variables influencing the IPv6 protocol settings
\item /proc/sys/net/netfilter - contains variables influencing the netfilter settings, not discussed in this thesis
\item /proc/sys/net/unix - contains variables influencing communication over unix sockets, not discussed in this thesis
\item /proc/sys/net/core - contains variables influencing low-level networking settings, including parameters of NAPI, Low Latency Sockets, etc.
\end{itemize}

The most important setting for the routing performance measurements of the Linux kernel is IPv4 forwarding.
It can be enabled by writing "1" to the /proc/sys/net/ipv4/ip\_forward file.
This variable is special - its change resets all IPv4 configuration parameters to their default state~\cite{kernel-doc-ip-sysctl}.
The /proc/sys/net/ipv4/conf/{\it{ifname}}/forwarding file can be used
to further selectively enable or disable forwarding on a particular interface.
Historically, some of the files in /proc/sys/net/ipv4 also influence settings of L4 protocols,
such as memory limits, TCP Timestamping, Selective ACKs, etc.
Although these files are located in the {\it{ipv4}} subdirectory, the L4 settings are independent on the underlying protocol.
Most of the Layer~4 settings are auto-tuned by the kernel itself and
their description is outside the scope of this thesis~\cite{linux-kernel-networking}.

Files in the {\it{ipv6}} directory influence the IPv6 protocol settings only.
The IPv6 protocol is disabled in CentOS 7 on all interfaces by default.
To enable the IPv6 protocol on all interfaces,
the variable accessible via /proc/sys/net/ipv6/conf/all/disable\_ipv6 must be changed to "0".
Similarly, the /proc/sys/net/ipv6/conf/all/forwarding variable must changed to "1" to enable IPv6 forwarding on all interfaces.
Both settings can be changed on per-interface basis as well.

The /proc/sys/net/ipv4/route.max\_size sets the maximum number of IPv4 routes allowed in the kernel.
This is 2~147~483~647 by default in CentOS~7, which is enough for a full BGP table,
which contains approx.~538~000 prefixes at the time of writing~\cite{bgp-analysis-reports}.
The /proc/sys/net/ipv6/route.max\_size sets the maximum number of IPv6 routes allowed in the kernel.
This is 4096 by default in CentOS~7, which must be raised for the measurements involving IPv6 BGP routes.
The number of IPv6 prefixes announced in the Internet is approx.~22~000 at the time of writing~\cite{bgp-analysis-reports}.

The source IPv4 address validation is enabled by default in CentOS 7.
This feature is called Reverse path filtering ({\it{rp\_filter}}) in the Linux kernel and it prevents IP spoofing.
However, it introduces additional processing and thus it should be disabled during the experiments.
The {\it{rp\_filter}} can be disabled on a particular interface
by writing "0" to /proc/sys/net/ipv4/conf/{\it{ifname}}/rp\_filter~\cite{kernel-doc-ip-sysctl}.
The {\it{rp\_filter}} for IPv6 is implemented in the netfilter subsystem of the Linux kernel and
thus it can be configured by ip6tables~\cite{kernel-source}.

Files in core directory provide access to low-level variables of the networking code.
There are two parameters that influence NAPI processing.
The /proc/sys/net/core/dev\_weight file sets the maximum number of packets that a single device
can feed to the kernel in its {\it{poll()}} function.
The default value is 64 in CentOS 7.
This value can be increased to allow the device to feed more packets at once.
However, most of the drivers provide their own limit which cannot be overwritten unless the code of the driver is changed.
This is the case of the mlx4 driver as well, which imposes the limit to 64 packets~\cite{kernel-source}.
The /proc/sys/net/core/netdev\_budget file sets the
maximum number of packets taken from all interfaces by a single {\it{net\_rx\_action()}} run.
The interfaces which are registered to polling are
probed in a round-robin manner, as described in subsection~\ref{sub:linux-ingress-napi}.
To allow the kernel to spend more time on packet processing, the {\it{netdev\_budget}} value can be increased.

The proc filesystem further provides access to the IRQ settings and statistics.
The /proc/interrupts file exports a table of all registered interrupts and their respective counters for each CPU.
Each registered interrupt has its own IRQ number.
On a multiprocessor system the interrupt can be served by any of the present CPU if the physical bus supports it.
This is the case of the PCI-Express MSI-X feature, which allows to deliver an interrupt to a specified CPU,
as described in section~\ref{sec:40gbe-throughput}.
%TODO QPI

Section~\ref{sec:linux-scaling} described how a network adapter with multiple queues
allows to scale the network traffic processing on a multi-processor system.
Each queue has assigned its own separate IRQ, thus it can be served by a particular CPU if configured properly.
The targeted CPU is defined using a mask written to the /proc/irq/{\it{NUMBER}}/smp\_affinity file,
or using a list of CPUs written to the /proc/irq/{\it{NUMBER}}/smp\_affinity\_list file.
The /proc/irq/default\_smp\_affinity file specifies a
default mask of CPUs for newly registered interrupts~\cite{kernel-doc-irq-affinity}.

The Linux kernel does not set the IRQ mapping automatically.
Instead, the user must configure it manually.
The {\it{irqbalance}} daemon reads the content of the /proc/interrupts file
and assigns the IRQ mappings according to the load.
The {\it{irqbalance}} daemon is part of the CentOS~7 and it is enabled by default.
While the {\it{irqbalance}} can introduce some performance advantage,
it does not take scaling or CPU placements (NUMA) into account~\cite{irqbalance-source}.

The CentOS~7 operating system further features the {\it{tuned}} utility in its default installation.
The {\it{tuned}} allows user to switch between user definable tuning profiles.
Several predefined profiles are already included, such as {\it{network-throughput}} or {\it{network-latency}}.
However, none of the profiles influences the low-level packet processing parameters described in chapter~\ref{chap:linux}.
Instead, the {\it{tuned}} focuses on L4 protocol parameters such as socket memory options~\cite{tuned}.


%=========================================================================
% (c) 2014, 2015 Josef Lusticky

\subsection{Sysfs settings}\label{sub:analysis-settings-sysfs}
Apart from the {\it{proc}} filesystem, the Linux kernel provides another virtual filesystem found under /sys in CentOS~7.
The {\it{sys}} filesystem exports information about loaded modules, including the parameters if a module takes any.
The /sys/modules/mxl4\_core/parameters directory contains parameters used by the mlx4\_core module.
The {\it{msi\_x}} parameter is set to 1 by default, which means attempt to use MSI-X.
The /sys/modules/mxl4\_en/parameters directory contains parameters used by the mlx4\_en module.
The {\it{udp\_rss}} parameter is set to 1 by default, which enables RSS for incoming UDP traffic.
There are more parameters taken by both modules, but they are not discussed in this thesis.
Further description of the parameters can be found
in the {\it{drivers/net/ethernet/mellanox/mlx4}} directory of the Linux kernel source code~\cite{kernel-source}.

Each network interface is represented by a symlink in the /sys/class/net/ directory.
The symlinks point to the corresponding network device, which is represented as a directory in {\it{sysfs}}.
The scaling mechanisms described in section~\ref{sec:linux-scaling} can be set
using the files exported in /sys/class/net/{\it{ifname}}/queues.
Each rx-{\it{xx}} subdirectory represents a single hardware receive queue.
The rx-{\it{xx}}/rps\_cpus file can be used to set a mask of CPUs serving interrupt requests from a particular hardware queue.
Since the Mellanox ConnectX-3 NIC supports RSS, the RPS feature is disabled by default - each rps\_cpus is set to 0.
Similary, the XPS feature can be configured via tx-{\it{xx}}/xps\_cpus.
The XPS configuration should be always checked to reflect the IRQ affinity mappings configured via proc.


%=========================================================================
% (c) 2014, 2015 Josef Lusticky

\subsection{Ethtool settings}\label{subsec:analysis-settings-ethtool}
Ethtool is a standard Linux utility for manipulating network drivers and hardware, particularly for
wired Ethernet devices.
Supported offload features can be displayed using ethtool -{}-show-offload {\it{devname}}.
Listing~\ref{lst:analysis-ethtool-offload} shows output of ethtool -{}-show-offload eth0,
where eth0 is the Mellanox ConnectX-3 adpater.

\begin{lstlisting}[caption={Output of ethtool -{}-show-offload for Mellanox ConnectX3 adapter},label={lst:analysis-ethtool-offload}]
rx-checksumming: on
tx-checksumming: on
	tx-checksum-ipv4: on
	tx-checksum-ip-generic: off [fixed]
	tx-checksum-ipv6: on
	tx-checksum-fcoe-crc: off [fixed]
	tx-checksum-sctp: off [fixed]
scatter-gather: on
	tx-scatter-gather: on
	tx-scatter-gather-fraglist: off [fixed]
tcp-segmentation-offload: on
	tx-tcp-segmentation: on
	tx-tcp-ecn-segmentation: off [fixed]
	tx-tcp6-segmentation: on
udp-fragmentation-offload: off [fixed]
generic-segmentation-offload: on
generic-receive-offload: on
large-receive-offload: off [fixed]
rx-vlan-offload: on [fixed]
tx-vlan-offload: on [fixed]
ntuple-filters: off [fixed]
receive-hashing: on
highdma: on [fixed]
rx-vlan-filter: on [fixed]
vlan-challenged: off [fixed]
tx-lockless: off [fixed]
netns-local: off [fixed]
tx-gso-robust: off [fixed]
tx-fcoe-segmentation: off [fixed]
tx-gre-segmentation: off [fixed]
tx-ipip-segmentation: off [fixed]
tx-sit-segmentation: off [fixed]
tx-udp_tnl-segmentation: off [fixed]
tx-mpls-segmentation: off [fixed]
fcoe-mtu: off [fixed]
tx-nocache-copy: on
loopback: off
rx-fcs: off [fixed]
rx-all: off [fixed]
tx-vlan-stag-hw-insert: off [fixed]
rx-vlan-stag-hw-parse: off [fixed]
rx-vlan-stag-filter: off [fixed]
\end{lstlisting}

Selected offload features can be enabled by ethtool -{}-offload {\it{devname}} {\it{feature}} on,
or disabled by ethtool -{}-offload {\it{devname}} {\it{feature}} off.
The features listed as [fixed] cannot be changed on this particular NIC.
The rest of the features can be changed, but the feature name differs when executing ethtool -{}-offload.
For example, the rx-checksumming feature is turned on by ethtool -{}-offload eth0 {\it{rx}} on.
Similarly, the scatter-gather feature is turned on by ethtool -{}-offload eth0 {\it{sg}} on.
For more information about using ethtool, the man page of ethool(8) should be consulted.

Listing~\ref{lst:analysis-ethtool-offload} shows that all supported offload features not marked as [fixed] are enabled.


%Interrupt coalescence tunning:
%-C eth0 rx-usecs 50

%Ring buffer tunning:
%-G eth0 rx 1024

%\subsection{Neighboring}
%The destination link layer address must be present in the ARP cache to avoid aditional delay of packet transmission.
%The description of the neighboring implementation is outside the scope of this thesis.
%The amount of ARP and Neighbor Discovery messages is negligible compared to the
%amount of the actual traffic and it is not considered by the measurements.
%However, their processing is not.


%MLX4 driver -  mlx4_en_add - rounddown_pow_of_two
% num_tx_rings_p_up , MLX4_EN_MAX_TX_RING_P_UP
% pocet_TX_rings = pocet_cpu * pocet_RX_rings


%TX,RX RING buffer sizes:
%Bufferbloat is a phenomenon in packet-switched networks generally,
%in which excess buffering of packets causes high latency and packet delay variation (also known as jitter),
%as well as reducing the overall network throughput.
%When a router device is configured to use excessively large buffers,
%even very high-speed networks can become practically unusable for many interactive applications like voice calls,
%chat, and even web surfing. %cite http://en.wikipedia.org/wiki/Network_scheduler


%Note that ip route show displays the main table.
%For displaying the local table, you should run ip route show table local~\cite{linux-kernel-networking}.
