%=========================================================================
% (c) 2014, 2015 Josef Lusticky

\subsection{CentOS~7 kernel compile-time configuration}
When the Linux kernel is compiled with support for symmetric multiprocessing
with the CONFIG\_SMP option and runs on a multiprocessor system, the code for receiving and
transmitting packets takes full advantage of that.
Every modern GNU/Linux distribution compiled for the AMD64 architecture has the option enabled,
including the CentOS~7~\cite{understanding-internals}.
Advanced networking features enabled in compile-time
does not have to be used in run-time.
For example, the MULTIPLE\_IP\_TABLES support is enabled in the CentOS 7 distribution kernel, however,
since the measurements presented in this thesis use no policy routing,
the simple FIB lookup principle described in section~\ref{sec:linux-routing} is still performed.

The CentOS~7 Linux kernel configuration has the CONFIG\_NO\_HZ\_FULL option enabled.
That is, the system uses adaptive ticks
and there are no regular interrupts from the timer
which would cause additonal delays during the packet processing~\cite{kernel-doc-nohz}.

Apart from compile-time options, the Linux kernel configuration can be changed in run-time.
The {\it{proc}} and {\it{sys}} filesystems provide access to the kernel variables that influence packet processing.
Tuning of these variables can provide a significant performance improvement when configured properly.
