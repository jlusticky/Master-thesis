%============================================================================
% tento soubor pouzijte jako zaklad
% (c) 2008 Michal Bidlo
% E-mail: bidlom AT fit vutbr cz
%============================================================================
% kodovaní: utf-8 (zmena prikazem iconv, recode nebo cstocs)
%----------------------------------------------------------------------------
% zpracování: make, make pdf, make desky, make clean
% připomínky posílejte na e-mail: bidlom AT fit.vutbr.cz
% vim: set syntax=tex encoding=latin2:
%============================================================================
\documentclass[english]{fitthesis} % odevzdani do wisu - odkazy, na ktere se da klikat
%\documentclass[english,print]{fitthesis} % pro tisk - na odkazy se neda klikat
%      \documentclass[english]{fitthesis}
% * Je-li prace psana v anglickem jazyce, je zapotrebi u tridy pouzit 
%   parametr english nasledovne:
%      \documentclass[english]{fitthesis}
% * Neprejete-li si vysazet na prvni strane dokumentu desky, zruste 
%   parametr cover

% zde zvolime kodovani, ve kterem je napsan text prace
% "latin2" pro iso8859-2 nebo "cp1250" pro windows-1250, "utf8" pro "utf-8"
%\usepackage{ucs}
\usepackage[czech,english]{babel}
\usepackage[utf8]{inputenc}
\usepackage[T1, IL2]{fontenc}
\usepackage{url}
%\usepackage{rotating}
\usepackage{listings}
%\lstset{language=C}
\DeclareUrlCommand\url{\def\UrlLeft{<}\def\UrlRight{>} \urlstyle{tt}}

%zde muzeme vlozit vlastni balicky
\usepackage{textcomp}
\usepackage[absolute]{textpos}
\usepackage{float}
\usepackage{color}
\definecolor{dkgreen}{rgb}{0,0.6,0}
\definecolor{gray}{rgb}{0.5,0.5,0.5}
\definecolor{mauve}{rgb}{0.58,0,0.82}

\lstset{ %
language=C,                % choose the language of the code
basicstyle=\footnotesize,       % the size of the fonts that are used for the code
numbers=none,                   % where to put the line-numbers
numberstyle=\footnotesize\color{gray},      % the size of the fonts that are used for the line-numbers
stepnumber=1,                   % the step between two line-numbers. If it is 1 each line will be numbered
numbersep=5pt,                  % how far the line-numbers are from the code
backgroundcolor=\color{white},  % choose the background color. You must add \usepackage{color}
showspaces=false,               % show spaces adding particular underscores
showstringspaces=false,         % underline spaces within strings
showtabs=false,                 % show tabs within strings adding particular underscores
frame=single,           % adds a frame around the code
rulecolor=\color{black},        % if not set, the frame-color may be changed on line-breaks within not-black text (e.g. commens (green here))
tabsize=2,          % sets default tabsize to 2 spaces
captionpos=b,           % sets the caption-position to bottom
breaklines=true,        % sets automatic line breaking
breakatwhitespace=false,    % sets if automatic breaks should only happen at whitespace
%title=\lstname,                   % show the filename of files included with \lstinputlisting; also try caption instead of title
escapeinside={\%*}{*)},          % if you want to add a comment within your code
keywordstyle=\color{blue},          % keyword style
commentstyle=\color{dkgreen},       % comment style
stringstyle=\color{mauve},         % string literal style
morekeywords={*,...}               % if you want to add more keywords to the set
}

% =======================================================================
% balíček "hyperref" vytváří klikací odkazy v pdf, pokud tedy použijeme pdflatex
% problém je, že balíček hyperref musí být uveden jako poslední, takže nemůže
% být v šabloně
\ifWis
\ifx\pdfoutput\undefined % nejedeme pod pdflatexem
\else
  \usepackage{color}
  \usepackage[unicode,colorlinks,hyperindex,plainpages=false,pdftex]{hyperref}
  \definecolor{links}{rgb}{0.4,0.5,0}
  \definecolor{anchors}{rgb}{1,0,0}
  \def\AnchorColor{anchors}
  \def\LinkColor{links}
  \def\pdfBorderAttrs{/Border [0 0 0] }  % bez okrajů kolem odkazů
  \pdfcompresslevel=9
\fi
\fi

%Informace o praci/projektu
%---------------------------------------------------------------------------
\projectinfo{
  %Prace
  project=DP,            %typ prace BP/SP/DP/DR
  year=2015,             %rok
  date=\today,           %datum odevzdani
  %Nazev prace
  title.cs={40GbE směrovač pro operační systém GNU/Linux},  %nazev prace v cestine
  title.en={Towards 40GbE GNU/Linux Router}, %nazev prace v anglictine
  %Autor
  author={Josef Luštický},   %jmeno prijmeni autora
  author.title.p=Bc., %titul pred jmenem (nepovinne)
  %author.title.a=PhD, %titul za jmenem (nepovinne)
  %Zadani
  task=fig/task.jpg,
  %Ustav
  department=UIFS, % doplnte prislusnou zkratku: UPSY/UIFS/UITS/UPGM
  %Skolitel
  supervisor= Matěj Grégr, %jmeno prijmeni skolitele
  supervisor.title.p=Ing.,   %titul pred jmenem (nepovinne)
  %supervisor.title.a={Ph.D.},    %titul za jmenem (nepovinne)
  %Klicova slova, abstrakty, prohlaseni a podekovani je mozne definovat 
  %bud pomoci nasledujicich parametru nebo pomoci vyhrazenych maker (viz dale)
  %===========================================================================
  %Klicova slova
  keywords.cs={GNU, Linux, ethernet, směrovač, software, IP, síť, měření, propustnost, operační systém}, %klicova slova v ceskem jazyce
  keywords.en={GNU, Linux, ethernet, router, software, IP, network, measurement, throughput, operating system}, %klicova slova v anglickem jazyce
  %Abstract
  abstract.cs={Účelem této práce je popis protokolu 40Gb Ethernet, popis směrovacího procesu v jádře Linux
  a navrhnout a provést testování výkonnosti směrování se síťovým adaptérem pro 40Gb Ethernet.
  Výsledky a nastavení pro získání maximální výkonnosti směrování jsou dále popsány v této práci.
  }, % abstrakt v ceskem jazyce
  abstract.en={The purpose of this thesis is to describe 40Gb Ethernet, describe routing process in the Linux kernel
  and to design and perform benchmark of routing performance with a 40Gb Ethernet network interface card.
  The results and system settings for achieving maximum routing performance are further described in the thesis.
  }, % abstrakt v anglickem jazyce
  %Prohlaseni
  declaration={Prohlašuji, že jsem tuto diplomovou práci vypracoval samostatně pod vedením pana Ing. Matěje Grégra.
  },
  %Podekovani (nepovinne)
  acknowledgment={Děkuji vedoucímu práce Ing. Matějovi Grégrovi z FIT VUT za poskytnutí praktických rad,
  vybaveného pracovního místa v laboratoři a pomoc při sestavování hardwaru.
  Děkuji Ing. Viktorovi Pušovi a Ing. Štěpánovi Friedlovi ze sdružení CESNET a projektu Liberouter.org
  za poskytnutí měřícího vybavení.
  Děkuji Ing. Pavlovi Kislingerovi z VUT za poskytnutí výkonného serveru.} % nepovinne
}

%Abstrakt (cesky, anglicky)
%\abstract[cs]{Do tohoto odstavce bude zapsán výtah (abstrakt) práce v českém jazyce.}
%\abstract[en]{Do tohoto odstavce bude zapsán výtah (abstrakt) práce v anglickém jazyce.}

%Klicova slova (cesky, anglicky)
%\keywords[cs]{Sem budou zapsána jednotlivá klíčová slova v českém jazyce, oddělená čárkami.}
%\keywords[en]{Sem budou zapsána jednotlivá klíčová slova v anglickém jazyce, oddělená čárkami.}

%Prohlaseni
%\declaration{Prohlašuji, že jsem tuto bakalářskou práci vypracoval samostatně pod vedením pana X...
%Další informace mi poskytli...
%Uvedl jsem všechny literární prameny a publikace, ze kterých jsem čerpal.}

%Podekovani (nepovinne)
%\acknowledgment{V této sekci je možno uvést poděkování vedoucímu práce a těm, kteří poskytli odbornou pomoc
%(externí zadavatel, konzultant, apod.).}

\begin{document}
  % Vysazeni titulnich stran
  % ----------------------------------------------
  \maketitle

\mktxt{16}{\textbf{Rozšířený abstrakt}}
Diplomová práce popisuje standard IEEE 802.3ba z roku 2010, který definuje protokol 40 a 100~Gigabit Ethernet,
konkrétně se v kapitole~\ref{chap:40gbe} věnuje protokolu 40~Gb Ethernet, jehož aplikace se dnes postupně dostává do praxe.
V současnosti je provoz tohoto protokolu možný pouze na optických spojích, nicméně plánovaný standard IEEE 802.3bq
počítá i~s~provozem 40~Gb Ethernetu po metalickém vedení.
40~Gb Ethernet zůstává do značné míry zpětně kompatibilní se staršími standardy Ethernetu, zejména
formát rámce zůstal zcela nezměněn.
Mezi počítači využívajícími protokol 40~Gb Ethernetu tak může být v~jednom směru posíláno
až přibližně 59 milionů rámců za sekundu o minimální velikosti 72 bajtů,
nebo až přibližně 4,6~GB přenášených dat za sekundu při posílání rámců o maximální velikosti 1526 bajtů.

Současný vývoj výkonu procesorů nestačí držet krok s narůstající
rychlostí komunikačních protokolů na linkové vrstvě ISO/OSI modelu.
Zatímco rychlost procesorů se zdvojnásobí přibližně jednou za 2 roky, rychlost linkových protokolů se zdvojnásobí za 18 měsíců.
Při 59 milionech rámců za sekundu je interval mezi dvěma po sobě příchozími rámci přibližně 16~nanosekund.
Za tuto dobu musí být systém schopen daný rámec korektně zpracovat, jinak hrozí jeho zahlcení dalším síťovým provozem.

Operační systém GNU/Linux se snaží držet krok se zrychlováním síťové komunikace pomocí množství optimalizací.
Kapitola~\ref{chap:linux} popisuje zpracování síťového provozu tímto operačním systémem se zaměřením na směrování IP paketů.
Síťový stack zodpovědný za zpracování paketů a směrování je implementován v jádře opračního systému Linux.
Síťový stack jádra Linux používá pro reprezentaci síťových paketů
strukturu {\it{sk\_buff}} ve všech vrstvách síťového stacku.
Při přechodu mezi jednotlivými vrstvami je předáván pouze ukazatel na~tuto strukturu
s patřičným pozměněním hlaviček struktury dle dané vrstvy.

Při zpracování paketu vrstvou zodpovědnou za síťový protokol IP dochází ke zpracování na základě rozhodnutí směrovacího subsystému.
Směrovací subsystém využívá interní směrovací databázi (Forwarding Information Base) k rozhodnutí o
následující funkci, která bude daný paket zpracovávat.
Forwarding Information Base je v Linuxu implentovaná pomocí struktury Trie.
Linux využívá algoritmus Longest Prefix Match k prohledání této struktury.
Výsledkem rozhodnotí může být zahození paketu, předání funkci k lokálnímu doručení nebo
předání fuknci {\it{ip\_forward}}, čímž dochází ke směrování daného paketu.
Obdobně funguje i směrování paketů protokolu IPv6.

Kromě tohoto zpracování, které je specifické pro směrování IP paketů, je zpracování síťového provozu
spojeno s další režií jako je oznámení o příchozím paketu pomocí přeřušení,
počítání kontrolních součtů, přiřazení do front atd.
O snížení této režie se snaží jak vývojáři jádra pomocí mechanismů jako je NAPI nebo Generic Receive Offload,
tak výrobci síťových karet pomocí hardwarového počítání kontrolních součtů nebo podporou vícefrontového zpracování.
Právě podpora více front v síťových kartách umožnuje společně s vlastnostmi sběrnice PCI-Express
distribuování zpracování paketů na více procesorech a~tím škálování síťové propustnosti.
Škálování je hlavním tématem současného vývoje a má největší vliv na celkovou propustnost systému.

Pro účely měření byla využita síťová karta Mellanox ConnectX-3 EN se 2 fyzickými porty
a hardwarový generátor provozu Spirent.
Kapitola~\ref{chap:analysis} popisuje jakým způsobem lze testování s poskytnutým hardwarem provést
a jaká je možnost konfigurace parametrů jádra Linuxu s ohledem na popsané zpracování paketů v předchozí kapitole.

Kapitola~\ref{chap:setup} popisuje postup zapojení a zprovoznění testovací sítě pro účely měření.
Instalovaný operační systém je CentOS~7 s jádrem verze 3.10.0-123.20.1. %TODO a dále také upstream jádro 3.19.2.
Dále je zde popsána konfigurace operačního systému, instalace nového firmware síťové karty a konfigurace hardwarového generátoru
paketů Spirent.

V kapitole~\ref{chap:measurements} jsou prezentovány výsledky měření a vliv jednotlivých konfiguračních možností
na výkon směrování paketů v jádře Linux.
Tyto výsledky jsou dále stručně komentovány v kapitole~\ref{chap:conclusion}, kde jsou také shrnuty nabyté poznatky
a identifikovány hlavní problémy zamezující lepší propustnosti.

Součástí diplomové práce jsou i přílohy s návodem importování internetových směrovacích záznamů z protokolu BGP
a stručný souhrn kroků pro dosažení maximálního výkonu směrování v operačním systému GNU/Linux.

  % Obsah
  % ----------------------------------------------
  \setcounter{tocdepth}{1}
  \tableofcontents
  
  % Seznam obrazku a tabulek (pokud prace obsahuje velke mnozstvi obrazku, tak se to hodi)
  % \listoffigures
  % \listoftables 

  % Text prace
  % ----------------------------------------------
  %=========================================================================
% (c) 2014, 2015 Josef Lusticky

%=========================================================================
% (c) 2014, 2015 Josef Lusticky

\chapter{Introduction}
The growth of Ethernet from 10 Mbit/s to 10 Gbit/s has already surpassed
the growth of microprocessor performance.
The 40~Gigabit Ethernet makes the performance gap even larger, but
it is still the original Ethernet underneath - an old technology
with a lot of compatibility issues for high-speed networking.
The recent 40 and 100~Gigabit Ethernet standard opens doors to
high-speed networking, but it requires other parts of the network to scale within.

The GNU/Linux operating system is used in a wide range of computers interconnected with high-speed Ethernet.
An important task of the Linux network stack is to forward traffic.
This is relevant especially when discussing core routers, which operate in the Internet backbone.
Forwarding occurs on Layer~3 of the ISO/OSI network model.
The performance of a software-based solution that uses Linux, cannot compete
with commercial products that can count on the help of specialised hardware.
However, various stack bypass solutions have shown, that the Linux kernel is not using
the CPU optimally.

The purpose of the thesis is to provide a comprehensive performance analysis of the Linux kernel
in packet forwarding.
The 40~Gigabit Ethernet protocol provides
frame rates of up to 59~million packets per second
and throughput of 4.6~GB per second.
Such speed can easily burden the CPU with a large amount of TCP/IP protocol processing required.

Apart from the 40~Gigabit Ethernet protocol itself,
the packet processing in the Linux kernel is described in the thesis.
Since the emerge of 100~Mbps Ethernet, the Linux kernel
engineers have been optimising the network stack towards high-speed packet processing.
Hardware vendors have also made various optimisations towards high-speed packet processing -
network interface cards support offloads, processors direct cache access, etc.





A high-end server with a 40~GbE network interface card
and 2 Intel Xeon CPUs was setup to measure the routing performance of the Linux kernel.
The measurements presented in this thesis demonstrate perfomance influences of various system settings
such as Reverse path filtering, Netfilter or SELinux.
Apart from ,
the routing perfomance with imported routes from the Internet BGP protocol was tested.
At the time of writing, there are approx.~538~000 routes announced in the public BGP,
which leads to expensive software lookups in the Forwarding Information Base of the Linux kernel.


Spirent hardware packet generator was used to measure the routing perfomance of the Linux kernel.



Measuring performance of the software IP routing using GNU/Linux-based operating system on 40~Gigabit Ethernet
can reveal bottlenecks that need to be eliminated
on the way to a full-speed 40~Gigabit TCP/IP processing.


if the system processes packets on Layer~3 fast enough,
next step is to optimise TCP and higher Layers of the network stack.





The performance of packet processing in the GNU/Linux operating
system 
scaling mechanisms.




system settings for achieving maximum routing performance.




Another bottleneck is the TCP/IP stack being processed at a rate less than the network speed.

The processing of TCP/IP over Ethernet is traditionally accomplished by software running on the CPUs of the server.





The server GNU/Linux operating system
Spirent hardware packet generator.



%=========================================================================
% (c) 2014, 2015 Josef Lusticky

\chapter{40 Gigabit Ethernet}

The growth of Ethernet from 10 Mbit/s to 10 Gbit/s has surpassed
the growth of microprocessor performance~\cite{10gea-toe}.
A fundamental obstacle to improving network performance is that servers were designed
for computing rather than input and output (I/O).
The Internet revolution has drastically changed server requirements,
and I/O is becoming a major bottleneck in delivering high-speed computing.
The main reason for the bottleneck is the TCP/IP stack being processed at a rate less than the network speed.
The processing of TCP/IP over Ethernet is traditionally accomplished by software running on the CPUs of the server.

As network connections scale beyond Gigabit Ethernet speeds,
the CPU becomes burdened with the large amount of TCP/IP protocol processing required.
Reassembling out-of-order packets, resource-intensive memory copies, and interrupts put a tremendous load on the host CPU.
In high-speed networks, the CPU has to dedicate more processing to handle the network traffic than to the applications it is running.
A rough estimate of the CPU processing required to handle a given Ethernet link speed is,
for every one bit per second of network data processed, one hertz of CPU processing is required.
This general rule of thumb was first stated by PC Magazine in the mid 1990’s,
and is still used as a rule of thumb today~\cite{10gea-toe}.





%Previous Ethernet versions could use standard Cat 6 copper cable and RJ-45 connectors , %~\cite{10gbase-t}
%whereas 40 Gigabit runs on Quad Small Form Factor Pluggable (QSPF) - a high-density fiber connector with 12 strands of fiber.
%http://searchdatacenter.techtarget.com/feature/40-GbE-technology-Hurry-up-and-wait


The channel layout is shown in figure~\ref{fig:40gbit-ethernet-layout}.

\begin{figure}
	\centering
	\includegraphics[width=13cm,keepaspectratio]{fig/ethernet-layout.jpg}
	\caption{IEEE 802.3 Ethernet Channel Layout (source: Cisco Systems Inc.)}
	\label{fig:40gbit-ethernet-layout}
	\bigskip
\end{figure}


40Gbit Ethernet is still Etherneth underneath - it is an old technology
with a lot of compatibility issues for high speed networking~\cite{jls2009-gro}.
The biggest of them being the 1500-byte maximum transfer unit (MTU).
There are some extensions to transfer larger frames, known as Jumbo frames,
but IEEE has determined they will not support or define Jumbo frames
due to concerns around vendor and equipment interoperability~\cite{ethernetalliance-jumbo-frames}.


a 10G network link running at full speed will be transferring over 700 000 packets per seconds

Most connections of interest go across the Internet, and those are all bound by the lowest MTU in the entire path
and sometimes that MTU is even less than 1500 bytes.
So, while Jumbo frames might work well for local networks, there is still a limit of 1500 bytes on the wider Internet~\cite{jls2009-gro}.



%=========================================================================
% (c) 2014, 2015 Josef Lusticky

\chapter{Linux Kernel Packet Processing}

Linux version 3.10 consists of nearly 17 million lines of code.
http://www.cnet.com/news/linux-development-by-the-numbers-big-and-getting-bigger/

Processing packets in the Linux kernel.
There are two modes of processing packets from NIC.
The traditional way is interrupt-driven - each incomming packet is an
asynchronous event which causes interrupt.

The poll method introduced in the New API (NAPI) - only the first incomming packet causes interrupt
and other packets are polled.
NAPI is designed to improve the performance of high-speed networking.
The driver using NAPI provides the poll function to the kernel for reading received packets.

Most of the new drivers support this feature.
When under a heavy load, the packets are lost because of not enough space in NIC buffer.
The packets to be lost are not fed into the network stack, so they take no CPU time.
http://www.haifux.org/lectures/172/netLec.pdf


TCP Segment Offload - kernel is not dealing with segmenting.
ethtool -K eth1 tso on
ethtool -k eth1


\section{NAPI}

NAPI is a proven (www.cyberus.ca/~hadi/usenix-paper.tgz) technique
to improve network performance on Linux.
http://lwn.net/2002/0321/a/napi-howto.php3
---

NAPI ("New API", though it is not so new anymore, since linux kernel 2.6) is an extension to the device driver packet processing framework, which is designed to improve the performance of high-speed networking~\cite{linux-foundation-napi}.

NAPI works through:

Interrupt mitigation (coalescing)
 Improves throughput but degrades latency.
    High-speed networking can create thousands of interrupts per second, all of which tell the system something it already knew: it has lots of packets to process. NAPI allows drivers to run with (some) interrupts disabled during times of high traffic, with a corresponding decrease in system load.

Packet throttling 
    When the system is overwhelmed and must drop packets, it's better if those packets are disposed of before much effort goes into processing them. NAPI-compliant drivers can often cause packets to be dropped in the network adaptor itself, before the kernel sees them at all.

More careful packet treatment, with special care taken to avoid reordering packets. Out-of-order packets can be a significant performance bottleneck. 


NAPI additions to the kernel do not break backward compatibility and drivers may still process completions directly in interrupt context if necessary.

http://lwn.net/Articles/30107/

-----


NAPI is an interrupt mitigation mechanism used with network devices. When network traffic is heavy, the kernel can safely predict that incoming packets will be available anytime it gets around to looking, so there is no need to have the adapter interrupting it (possibly thousands of times per second) to tell it about those packets. So a NAPI-compliant driver will turn off the packet receive interrupt and provide a poll() method to the kernel. When the kernel is ready to deal with more packets, poll() will be called with a maximum number of packets it is allowed to feed into the kernel; it should process up to that many packets and quit.
http://lwn.net/Articles/214457/

-----

Network interfaces, like most reasonable peripheral devices, are capable of interrupting the CPU whenever a packet arrives. But even a moderately busy interface can handle hundreds or thousands of packets per second; per-packet interrupts would quickly overwhelm the processor with interrupt-handling work, leaving little time for getting useful tasks done. So most interface drivers will disable the per-packet interrupt when the traffic level is high enough and, with cooperation from the core networking stack, occasionally poll the device for new packets. There are a number of advantages to doing things this way: vast numbers of interrupts can be avoided, incoming packets can be more efficiently processed in batches, and, if packets must be dropped in response to load, they can be discarded in the interface before they ever hit the network stack. Polling is thus a win for almost all situations where there is any significant amount of traffic at all. 
http://lwn.net/Articles/551284/

---

NAPI is pretty good, but optimized for throughput

Latency is high by default (especially for Ethernet)
Jitter is unpredictable by default

netperf

http://www.linuxplumbersconf.org/2012/wp-content/uploads/2012/09/2012-lpc-Low-Latency-Sockets-slides-brandeburg.pdf

----

It is also possible to use NAPI in wireless 802.11 drivers.


%=========================================================================
% (c) 2014, 2015 Josef Lusticky

\chapter{Analysis}\label{chap:analysis}
Basically, there are three important parts needed to perform network benchmarks at full 40~Gbit speed -
a network interface card with 40~Gigabit Ethernet support, a server compatible with the card and a packet generator
capable of generating 40~Gbps network traffic.
A chosen distribution of the GNU/Linux operating system should contain a stable and
recent kernel that supports the 40~Gb Ethernet protocol, the network interface card
and the features described in chapter~\ref{chap:linux}.

%=========================================================================
% (c) 2014, 2015 Josef Lusticky

\section{Hardware equipment}\label{sec:analysis-hardware}
The network interface card used in the experiments is
Mellanox ConnectX-3 EN QSPF dual-port PCI-E 3.0 x8 MCX314A-BCBT~\cite{mellanox-product-brief}.
The card was provided by the Faculty of Information Technology, Brno University of Technology.
Mellanox ConnectX-3 EN is an adapter that can run 10~Gigabit Ethernet and 40~Gigabit Ethernet.
It also supports nonstandard 56~Gbps link speed when connected to Mellanox switches.
The card is PCI-Express 3.0 x8 compatible with support for previous PCI-Express versions.
Mellanox ConnectX-3 EN is a multiqueue NIC with MSI-X support up to 16 receive queues per port
featuring Receive Side Scaling with hashing support for both IPv4/IPv6 and TCP/UDP flows~\cite{mellanox-silicon, mellanox-user-manual}.
Figure~\ref{fig:setup-mlx-block-diagram} shows the block diagram of the network interface card.

\begin{figure}
	\centering
	\includegraphics[width=7.5cm,keepaspectratio]{fig/mlx-block-diagram.png}
	\caption{Mellanox ConnectX-3 EN block diagram (source:~\cite{mellanox-silicon})}
	\label{fig:setup-mlx-block-diagram}
	\bigskip
\end{figure}

The Mellanox NIC requires PCI-Express 3.0 x8 slot to take full advantage of its speed.
Brno University of Technology provided a server with
the Supermicro X10DRU-i+ motherboard, which
features PCI-Express 3.0 slots compatible with the Mellanox Connect-X 3 EN adapter~\cite{supermicro-board}.
The server is further equipped with two Intel Xeon E5-2660 v3 processors at 2200MHz with 10 physical cores per CPU
and 20 logical cores per CPU when Hyper-Threading is enabled.
Each CPU has 20~MB shared L3 cache and PCI Express 3.0 support with up to 40 lanes.
The processor features Direct Data I/O technology (also known as Direct Cache Access),
which optimises cache access for networking purposes by putting the ingress packtes directly to the CPU cache~\cite{intel-xeon-cpu}.

There are various software frameworks for high-speed packet generation,
such as pktgen\footnote{\url{https://www.kernel.org/doc/Documentation/networking/pktgen.txt}}
or Netmap\footnote{\url{http://info.iet.unipi.it/~luigi/netmap/}}.
Pktgen is an upstream component of the Linux kernel, but
at the time of writing it is not capable of generating even full 10~GbE frame rate~\cite{netmap}.
However, it can be combined with other frameworks for fast packet processing, such as
Intel's Data Plane Development Kit\footnote{\url{http://dpdk.org/}}.
Netmap is not an upstream component, but it provides patches to the Linux kernel.
Netmap claims to generate 14.88~million frames per second, which is a full frame rate of 10~Gigabit Ethernet~\cite{netmap}.
However, Netmap was not tested against 40~GbE full frame rate of 59.5~million frames per second,
as calculated in section~\ref{sec:40gbe-frame-rates}.
Although both frameworks seem promising, their benchmarking and description
are outside the scope of this thesis.
Moreover, to perform the measurements with a software-based packet generator,
another GNU/Linux server and a 40~GbE NIC is needed.

Another solution is to use a hardware-based packet generator such as Spirent~\cite{spirent}.
With kind permission of CESNET, the Czech national research and education network operator,
the Spirent SPT-3U equipped with a combined 100Gb / 2x40Gb Ethernet module was used to perform the measurements.
The Spirent packet generator supports generation of custom Layer 2-7 traffic, custom frame length and various
predefined traffic patterns with variable frame length called Internet Mix (iMix).
These patterns represent a typical distribution of frame lengths found in the Internet traffic
and they can be further customised.
Spirent SPT-3U further supports configuration of custom frame rates and bandwidth use~\cite{spirent}.


%=========================================================================
% (c) 2014, 2015 Josef Lusticky

\section{Software and firmware}
Base CentOS 7 was installed on the server.
The operating system features Linux kernel based on version 3.10 -
the installed version is 3.10.0-123.20.1.el7.x86\_64.
The operating system was updated with all updates avaliable as of 20th March 2015.
The upstream kernel version 3.19.2 was additionally installed from the ELRepo repository~\cite{elrepo-kernel-ml}.

The Linux kernel detects the Mellanox ConnectX-3 EN card automatically and loads the mlx4\_core and mlx4\_en module.
The mlx4\_core module prints the detected PCI-Express link parameters to the kernel's message buffer.
The buffer can be viewed using the dmesg command and its partial output is shown bellow:
\begin{lstlisting}[language=TeX]
mlx4_core 0000:06:00.0: PCIe link speed is 8.0GT/s, device supports 8.0GT/s
mlx4_core 0000:06:00.0: PCIe link width is x8, device supports x8
\end{lstlisting}
The mlx4\_core module further registers interrupts and prints the assigned IRQ numbers for each queue
to the kernel's message buffer:
\begin{lstlisting}[language=TeX]
mlx4_core 0000:06:00.0: irq 61 for MSI/MSI-X
mlx4_core 0000:06:00.0: irq 62 for MSI/MSI-X
...
mlx4_core 0000:06:00.0: irq 90 for MSI/MSI-X
\end{lstlisting}

The driver uses either MSI or MSI-X feature of the PCI-Express bus, as described in section~\ref{sec:40gbe-throughput}.
The MSI-X feature is used automatically if the system supports it, otherwise the adapter uses MSI.
The {\it{lspci -vv}} command can be used to check whether MSI-X is used -
the MSI-X capability is followed by an Enable flag which is followed with either "+" (enabled)
or "-" (disabled).
Listing~\ref{lst:setup-lspci} shows partial output of lspci for the Mellanox ConnectX-3 EN adapter.
The system supports MSI-X and the adapter is configured to use it.
\begin{lstlisting}[language=TeX,label={lst:setup-lspci},caption={Partial output of lspci -vv for Mellanox ConnectX-3 EN}]
06:00.0 Ethernet controller: Mellanox Technologies MT27500 Family [ConnectX-3]
		...
		Capabilities: [9c] MSI-X: Enable+ Count=128 Masked-
				...
				LnkCap: Port #8, Speed 8GT/s, Width x8, ASPM L0s, Exit Latency L0s unlimited, L1 unlimited
				...
\end{lstlisting}

Apart from the NIC driver, which is part of kernel package, the Mellanox ConnectX-3 adpater uses its own firmware.
The firmware was updated to version 2.32.5100, which is the latest version available as of 10th January 2015.
The firmware update procedure is described in appendix~\ref{app:firmware}.

%By default, the driver uses adaptive interrupt moderation for the receive path,
%which adjusts the moderation time to the traffic pattern~\cite{mellanox-user-manual}.


%=========================================================================
% (c) 2014, 2015 Josef Lusticky

\section{Benchmarking methodology}\label{sec:analysis-metodology}
Procedures described by RFC~2544 can be used to measure routing performance of the Linux kernel.
RFC~2544 specifies the benchmarking methodology for network interconnect devices.
The ideal way to implement the series of tests described in RFC~2544 is to use a tester
with both transmitting and receiving ports.
Connections are made from the sending ports of the tester to the receiving ports of the
device under test (DUT) and from the sending ports of the DUT back to the tester~\cite{rfc2544}.
Figure~\ref{fig:analysis-rfc2544} shows the test implementation.

Since the tester both sends the test traffic and receives
it back, after the traffic has been forwarded by the DUT, the tester
can easily determine if all of the transmitted packets were received~\cite{rfc2544}.
The Spirent TestCenter Application provides statistics about transmitted and received frames,
which can be used for this purpose.
\begin{figure}
	\centering
	\includegraphics[width=9cm,keepaspectratio]{fig/rfc2544.png}
	\caption{RFC2544 test implementation (source:~\cite{rfc2544})}
	\label{fig:analysis-rfc2544}
\end{figure}

\input{analysis/metodology-generation.tex}

\input{analysis/metodology-collection.tex}


%=========================================================================
% (c) 2014, 2015 Josef Lusticky

\section{Software settings}\label{sec:analysis-settings}
The CentOS 7 operating system features various components that influence forwarding performance.
To measure a pure routing performance of the Linux kernel,
the Netfilter and SELinux components should be disabled.
If disabling the netfilter is not appropriate, the iptables utility must be used to configure netfilter to allow forwarding,
because the default rules do not allow packet forwarding.
SELinux in the CentOS~7 operating system uses enforcing policy by default.
Similarly to netfilter, the SELinux component should be disabled to prevent performance decrease.
Influence of both the components on forwarding performance can be measured.

The Linux kernel features dynamic CPU frequency scaling.
The CPUFreq governors are policies that decide what frequency should be used.
The CPUfreq governor {\it{performance}} should be used during the measurements, as it sets the CPU statically to the
highest frequency avaliable~\cite{cpufreq-governors}. %, which is 2.6~GHz in case of Intel Xeon E5-2630 v2.

Packet processing by the Linux kernel can be configured in different ways.
To change the Linux kernel compile-time configuration, the kernel must be recompiled.
The CentOS~7 kernel provides a fair amount of features that could break existing setups when disabled.
The default kernel compile-time configuration does not have to provide the best routing performance,
however, it is usually used in most scenarious and hence its benchmark results are of interests for most people.

When the Linux kernel is compiled with support for symmetric multiprocessing
using the CONFIG\_SMP option and runs on a multiprocessor system, the code for receiving and
transmitting packets takes full advantage of that.
Every modern GNU/Linux distribution compiled for the AMD64 architecture has the option enabled,
including the CentOS~7~\cite{understanding-internals}.

Advanced features configured during compile-time of the Linux kernel introduce
just a negligible overhead when they are not used.
For example, the MULTIPLE\_IP\_TABLES support is enabled in the CentOS 7 distribution kernel, however,
since the measurements presented in this thesis use no policy routing,
the FIB lookup principle described in section~\ref{sec:linux-routing} is still performed.

Apart from compile-time options, the Linux kernel configuration can be changed during run-time.
The proc and sys filesystems provide access to the kernel variables that influence packet processing.
Tuning of these variables can provide significant amount of routing performance improvement when configured properly.

\input{analysis/settings-procfs.tex}

\input{analysis/settings-sysfs.tex}

\input{analysis/settings-ethtool.tex}

%Note that ip route show displays the main table.
%For displaying the local table, you should run ip route show table local~\cite{linux-kernel-networking}.



%=========================================================================
% (c) 2014, 2015 Josef Lusticky

\chapter{Setup}\label{chap:setup}

%=========================================================================
% (c) 2014, 2015 Josef Lusticky

\section{Hardware equipment}\label{sec:analysis-hardware}
The network interface card used in the experiments is
Mellanox ConnectX-3 EN QSPF dual-port PCI-E 3.0 x8 MCX314A-BCBT~\cite{mellanox-product-brief}.
The card was provided by the Faculty of Information Technology, Brno University of Technology.
Mellanox ConnectX-3 EN is an adapter that can run 10~Gigabit Ethernet and 40~Gigabit Ethernet.
It also supports nonstandard 56~Gbps link speed when connected to Mellanox switches.
The card is PCI-Express 3.0 x8 compatible with support for previous PCI-Express versions.
Mellanox ConnectX-3 EN is a multiqueue NIC with MSI-X support up to 16 receive queues per port
featuring Receive Side Scaling with hashing support for both IPv4/IPv6 and TCP/UDP flows~\cite{mellanox-silicon, mellanox-user-manual}.
Figure~\ref{fig:setup-mlx-block-diagram} shows the block diagram of the network interface card.

\begin{figure}
	\centering
	\includegraphics[width=7.5cm,keepaspectratio]{fig/mlx-block-diagram.png}
	\caption{Mellanox ConnectX-3 EN block diagram (source:~\cite{mellanox-silicon})}
	\label{fig:setup-mlx-block-diagram}
	\bigskip
\end{figure}

The Mellanox NIC requires PCI-Express 3.0 x8 slot to take full advantage of its speed.
Brno University of Technology provided a server with
the Supermicro X10DRU-i+ motherboard, which
features PCI-Express 3.0 slots compatible with the Mellanox Connect-X 3 EN adapter~\cite{supermicro-board}.
The server is further equipped with two Intel Xeon E5-2660 v3 processors at 2200MHz with 10 physical cores per CPU
and 20 logical cores per CPU when Hyper-Threading is enabled.
Each CPU has 20~MB shared L3 cache and PCI Express 3.0 support with up to 40 lanes.
The processor features Direct Data I/O technology (also known as Direct Cache Access),
which optimises cache access for networking purposes by putting the ingress packtes directly to the CPU cache~\cite{intel-xeon-cpu}.

There are various software frameworks for high-speed packet generation,
such as pktgen\footnote{\url{https://www.kernel.org/doc/Documentation/networking/pktgen.txt}}
or Netmap\footnote{\url{http://info.iet.unipi.it/~luigi/netmap/}}.
Pktgen is an upstream component of the Linux kernel, but
at the time of writing it is not capable of generating even full 10~GbE frame rate~\cite{netmap}.
However, it can be combined with other frameworks for fast packet processing, such as
Intel's Data Plane Development Kit\footnote{\url{http://dpdk.org/}}.
Netmap is not an upstream component, but it provides patches to the Linux kernel.
Netmap claims to generate 14.88~million frames per second, which is a full frame rate of 10~Gigabit Ethernet~\cite{netmap}.
However, Netmap was not tested against 40~GbE full frame rate of 59.5~million frames per second,
as calculated in section~\ref{sec:40gbe-frame-rates}.
Although both frameworks seem promising, their benchmarking and description
are outside the scope of this thesis.
Moreover, to perform the measurements with a software-based packet generator,
another GNU/Linux server and a 40~GbE NIC is needed.

Another solution is to use a hardware-based packet generator such as Spirent~\cite{spirent}.
With kind permission of CESNET, the Czech national research and education network operator,
the Spirent SPT-3U equipped with a combined 100Gb / 2x40Gb Ethernet module was used to perform the measurements.
The Spirent packet generator supports generation of custom Layer 2-7 traffic, custom frame length and various
predefined traffic patterns with variable frame length called Internet Mix (iMix).
These patterns represent a typical distribution of frame lengths found in the Internet traffic
and they can be further customised.
Spirent SPT-3U further supports configuration of custom frame rates and bandwidth use~\cite{spirent}.


%=========================================================================
% (c) 2014, 2015 Josef Lusticky

\section{Networking}\label{sec:setup-networking}
Default installation of CentOS with the updates avaliable on 7th January 2015 installed,
including the distribution Linux kernel version 3.10.0-123.13.1.el7.x86\_64.


IPv4 addresses from 192.0.2.0/24 (TEST-NET-1) block were assigned~\cite{rfc5737}.
IPv6 addresses from 2001:db8::/32 range were assigned,
addresses within this block should not appear on the public Internet~\cite{rfc3849}.

Section%~\ref{}
described how the mlx4 drivers set up network interfaces.
In the measurements, IPv4 addresses 1.0.1.1 and 1.0.2.1 with 24-bit subnet mask were assigned to the interfaces.
%These subnets are not part of the BGP routes.
On the Spirent, the corresponding addresses 1.0.1.2 and 1.0.2.2 with 24-bit subnet mask were assigned.
Figure~\ref{fig:measurements-setup} shows the network scheme used for the measurements.
\begin{figure}[H]
	\centering
	\includegraphics[width=13.5cm,keepaspectratio]{fig/net-setup.png}
	\caption{Measurement setup}
	\label{fig:measurements-setup}
\end{figure}


Enable IPv4 packet forwarding:
echo 1 > /proc/sys/net/ipv4/ip\_forward


ip neigh add 1.0.0.2 lladdr f4:52:14:5e:6c:71 dev enp6s0d1
ip neigh add 2.0.0.2 lladdr f4:52:14:5e:6c:70 dev enp6s0

ip addr add 1.0.0.1/24 broadcast 1.0.0.255 dev enp6s0d1
ip addr add 2.0.0.1/24 broadcast 2.0.0.255 dev enp6s0


Load BGP routes:
\begin{lstlisting}
Basic info: size of leaf: 40 bytes, size of tnode: 40 bytes.
Main:
        Aver depth:     2.43
        Max depth:      8
        Leaves:         503308
        Prefixes:       538739
        Internal nodes: 114430
          1: 58725  2: 26171  3: 14808  4: 7316  5: 4239  6: 2103  7: 1065  8: 2  17: 1
        Pointers: 995798
Null ptrs: 378061
Total size: 61373  kB
\end{lstlisting}


%=========================================================================
% (c) 2014, 2015 Josef Lusticky

\section{Software and firmware}
Base CentOS 7 was installed on the server.
The operating system features Linux kernel based on version 3.10 -
the installed version is 3.10.0-123.20.1.el7.x86\_64.
The operating system was updated with all updates avaliable as of 20th March 2015.
The upstream kernel version 3.19.2 was additionally installed from the ELRepo repository~\cite{elrepo-kernel-ml}.

The Linux kernel detects the Mellanox ConnectX-3 EN card automatically and loads the mlx4\_core and mlx4\_en module.
The mlx4\_core module prints the detected PCI-Express link parameters to the kernel's message buffer.
The buffer can be viewed using the dmesg command and its partial output is shown bellow:
\begin{lstlisting}[language=TeX]
mlx4_core 0000:06:00.0: PCIe link speed is 8.0GT/s, device supports 8.0GT/s
mlx4_core 0000:06:00.0: PCIe link width is x8, device supports x8
\end{lstlisting}
The mlx4\_core module further registers interrupts and prints the assigned IRQ numbers for each queue
to the kernel's message buffer:
\begin{lstlisting}[language=TeX]
mlx4_core 0000:06:00.0: irq 61 for MSI/MSI-X
mlx4_core 0000:06:00.0: irq 62 for MSI/MSI-X
...
mlx4_core 0000:06:00.0: irq 90 for MSI/MSI-X
\end{lstlisting}

The driver uses either MSI or MSI-X feature of the PCI-Express bus, as described in section~\ref{sec:40gbe-throughput}.
The MSI-X feature is used automatically if the system supports it, otherwise the adapter uses MSI.
The {\it{lspci -vv}} command can be used to check whether MSI-X is used -
the MSI-X capability is followed by an Enable flag which is followed with either "+" (enabled)
or "-" (disabled).
Listing~\ref{lst:setup-lspci} shows partial output of lspci for the Mellanox ConnectX-3 EN adapter.
The system supports MSI-X and the adapter is configured to use it.
\begin{lstlisting}[language=TeX,label={lst:setup-lspci},caption={Partial output of lspci -vv for Mellanox ConnectX-3 EN}]
06:00.0 Ethernet controller: Mellanox Technologies MT27500 Family [ConnectX-3]
		...
		Capabilities: [9c] MSI-X: Enable+ Count=128 Masked-
				...
				LnkCap: Port #8, Speed 8GT/s, Width x8, ASPM L0s, Exit Latency L0s unlimited, L1 unlimited
				...
\end{lstlisting}

Apart from the NIC driver, which is part of kernel package, the Mellanox ConnectX-3 adpater uses its own firmware.
The firmware was updated to version 2.32.5100, which is the latest version available as of 10th January 2015.
The firmware update procedure is described in appendix~\ref{app:firmware}.

%By default, the driver uses adaptive interrupt moderation for the receive path,
%which adjusts the moderation time to the traffic pattern~\cite{mellanox-user-manual}.


\section{Software settings}
Change the scaling governor for each CPU present in the system to performance (default is powersave):
\begin{lstlisting}
echo performance | tee /sys/devices/system/cpu/cpu[0-9]*/cpufreq/scaling_governor
\end{lstlisting}
Disable SELinux by editing SELINUX variable in /etc/sysconfig/selinux
(default is enforcing) and reboot the computer to apply:
\begin{lstlisting}
vi /etc/sysconfig/selinux
	SELINUX=disabled
reboot
\end{lstlisting}
%To disable the netfilter completely execute systemctl stop firewalld.
%If disabling the netfilter is not appropriate, flush all chains iptables -F,
%delete all user-defined chains iptables -X (which can be checked again by iptables -L).
Disable netfilter completely:
\begin{lstlisting}[language=TeX]
systemctl stop firewalld
systemctl disable firewalld  # do not execute firewalld on boot
\end{lstlisting}
Enable IPv6 on all interfaces (or at least on the forwarding interfaces):
\begin{lstlisting}
echo 0 > /proc/sys/net/ipv6/conf/all/disable\_ipv6
\end{lstlisting}
Enable IPv4 forwarding:
\begin{lstlisting}
echo 1 > /proc/sys/net/ipv4/ip\_forward
\end{lstlisting}
Enable IPv6 forwarding on all interfaces (or at least on the forwarding interfaces):
\begin{lstlisting}
echo 1 > /proc/sys/net/ipv6/conf/all/forwarding
\end{lstlisting}

IPv4 neighbours:
\begin{lstlisting}
ip neigh add 192.0.2.2 lladdr 00:10:94:00:00:01 dev enp6s0d1
ip neigh add 192.0.2.6 lladdr 00:10:94:00:00:02 dev enp6s0
\end{lstlisting}
IPv6 neighbours:
\begin{lstlisting}
ip -6 neigh add 2001:db8:1::2 lladdr 00:10:94:00:00:03 dev enp6s0d1
ip -6 neigh add 2001:db8:2::6 lladdr 00:10:94:00:00:04 dev enp6s0
\end{lstlisting}

Assign IPv4 addresses:
\begin{lstlisting}
ip addr add 192.0.2.1/30 broadcast 192.0.2.3 dev enp6s0d1
ip addr add 192.0.2.5/30 broadcast 192.0.2.7 dev enp6s0
\end{lstlisting}
Assign IPv6 addresses:
\begin{lstlisting}
ip -6 addr add 2001:db8:1::1/64 dev enp6s0d1
ip -6 addr add 2001:db8:2::5/64 dev enp6s0
\end{lstlisting}

%Additional system settings for a particular measurement are described.


%=========================================================================
% (c) 2014, 2015 Josef Lusticky

\chapter{Measurements}\label{chap:measurements}
The results presented in this thesis are based on the setup and basic settings described in chapter~\ref{chap:setup}.
The CentOS~7 operating system
was installed on a server equipped with two Intel Xeon E5-2660 v3 CPUs
and Mellanox ConnectX-3 EN 40~Gbps ethernet adapter, as described in section~\ref{sec:analysis-hardware}.
Unless stated otherwise,
the bandwidth use is configured in frames per second with a unit of margin 50~000.

\section{CentOS~7 distribution kernel 3.10}
The CentOS~7 distribution kernel version 3.10.0-123.20.1.el7.x86\_64
was used in the measurements presented in this section.

	%=========================================================================
% (c) 2014, 2015 Josef Lusticky

% FINAL
\subsection{Measurement 1 - default configuration - single IPv4 flow}
The first measurement shows the routing perfomance with
IP addresses assigned, IP forwarding enabled and Netfilter rules flushed.

The IP addresses were assigned as described in section~\ref{sec:setup-server}.
The IP forwarding was enabled by echoing "1" to the /proc/sys/net/ipv4/ip\_forward file and the Netfilter rules
were flushed using {\it{iptables -F}}, because the default rules do not allow forwarding.
\\
\\
\begin{tabular}{ | l | l | l | l | }
\hline
Frame size & \% of link & bandwidth & frame rate \\
\hline
64     &  0.59\% & 0.24~Gb/s & 350~000 \\
594    &  4.30\% & 1.72~Gb/s & 350~000 \\
1518   & 10.77\% & 4.31~Gb/s & 350~000 \\
AMS-IX &  5.33\% & 2.13~Gb/s & 350~000 \\
\hline
\end{tabular}

	
	%=========================================================================
% (c) 2014, 2015 Josef Lusticky

% FINAL
\subsection{Measurement 2 - default configuration - 32 IPv4 flows}
Since the Linux kernel scaling mechanisms described in section~\ref{sec:linux-scaling}
are based on processing each flow by a different CPU,
the routing performance of the default configuration was tested against 32 IPv4 flows.
\\
\\
\begin{tabular}{ | l | l | l | l | }
\hline
Frame size & \% of link & bandwidth & frame rate \\
\hline
64     &  2.69\% &  1.08~Gb/s & 1~550~000 \\
594    & 19.65\% &  7.86~Gb/s & 1~600~000 \\
1518   & 49.22\% & 19.69~Gb/s & 1~650~000 \\
AMS-IX & 24.34\% &  9.74~Gb/s & 1~600~000 \\
\hline
\end{tabular}
\\
\\
As expected, the scaling mechanisms help to increase the routing performance of the Linux kernel.
The scaling mechanisms perform better when forwarding larger frames.
This may be caused by differences in memory management allocations,
since the memory management is common for all CPUs present in the system.

	
	\input{measurements/el7/m3.tex}
	
	%=========================================================================
% (c) 2014, 2015 Josef Lusticky

% FINAL
\subsection{Measurement 4 - 32 independent IPv4 flows}
This measurement can be compared with Measurement~2, except that the {\it{irqbalance}} daemon is disabled.
Therefore, the IRQ mapping is left untouched in its default state.
\\
\\
\begin{tabular}{ | l | l | l | l | }
\hline
Frame size & \% of link & bandwidth & frame rate \\
\hline
64     &  1.26\% &  0.50~Gb/s & 750~000 \\
594    & 12.28\% &  4.91~Gb/s & 800~000 \\
1518   & 24.61\% &  9.84~Gb/s & 800~000 \\
AMS-IX & 15.22\% &  6.09~Gb/s & 800~000 \\
\hline
\end{tabular}
\\
\\
When forwarding IP packets from multiple IPv4 flows on a single CPU,
the routing performance of the Linux kernel drops by 20\% against forwarding a single IPv4 flow.
The following listing shows that
the packets are evenly distributed among all hardware queues.
However, the interrupts are not distributed among CPUs.
\begin{lstlisting}
      ... CPU8  CPU9   CPU10 CPU11 CPU12  ...
178:  ...    0     0  474701     0     0  ...  enp129s0-0
179:  ...    0     0       0     0     0  ...  enp129s0-1
180:  ...    0     0       0     0     0  ...  enp129s0-2
181:  ...    0     0       0     0     0  ...  enp129s0-3
182:  ...    0     0       0     0     0  ...  enp129s0-4
183:  ...    0     0       0     0     0  ...  enp129s0-5
184:  ...    0     0       0     0     0  ...  enp129s0-6
185:  ...    0     0       0     0     0  ...  enp129s0-7
186:  ...    0     0  317322     0     0  ...  enp129s0d1-0
187:  ...    0     0  317648     0     0  ...  enp129s0d1-1
188:  ...    0     0  317231     0     0  ...  enp129s0d1-2
189:  ...    0     0  317384     0     0  ...  enp129s0d1-3
190:  ...    0     0  317114     0     0  ...  enp129s0d1-4
191:  ...    0     0  317291     0     0  ...  enp129s0d1-5
192:  ...    0     0  317190     0     0  ...  enp129s0d1-6
193:  ...    0     0  317964     0     0  ...  enp129s0d1-7
\end{lstlisting}
Each receive queue triggers roughly the same number of interrupts as in the previous measurement,
but overall the NIC triggers much more interrupts.
The kernel spends more time on running the interrupt service routine code.
Apart from servicing interrupts, the kernel must fetch the packets from different ingress queues,
which in turn may need additional locking.
\\
The following listing shows partial output of {\it{perf}}:
\begin{lstlisting}
perf top -C 10
  12.07%  [kernel]  [k] _raw_spin_lock
   8.68%  [kernel]  [k] fib_table_lookup
   5.01%  [kernel]  [k] mlx4_en_xmit
   4.63%  [kernel]  [k] mlx4_en_process_rx_cq
   3.64%  [kernel]  [k] __netif_receive_skb_core
   3.49%  [kernel]  [k] memcpy
   3.08%  [kernel]  [k] irq_entries_start
   2.68%  [kernel]  [k] mlx4_eq_int
   2.33%  [kernel]  [k] mlx4_en_poll_tx_cq
   2.24%  [kernel]  [k] ip_route_input_noref
\end{lstlisting}
The kernel spends most of the time on locking and FIB table lookup.

	
	%=========================================================================
% (c) 2014, 2015 Josef Lusticky

% FINAL
\subsection{Measurement 5 - single IPv4 flow over QPI}
Instruct the kernel to use the CPU~9 for processing the RX interrupts.
The softirq and forwarding code runs on CPU~9.
\begin{lstlisting}[language=TeX]
for i in `seq 178 193` ; do echo 9 > /proc/irq/\$i/smp_affinity_list ;done
\end{lstlisting}
The CPU~9 is not directly connected to the PCI-Express link with the NIC,
so QPI links between CPUs are used.

\begin{tabular}{ | l | l | l | l | }
\hline
Frame size & \% of link & bandwidth & frame rate \\
\hline
64     &  1.09\% &  0.44~Gb/s & 650~000 \\
594    &  7.98\% &  3.19~Gb/s & 650~000 \\
1518   & 19.99\% &  8.00~Gb/s & 650~000 \\
AMS-IX &  9.89\% &  3.96~Gb/s & 650~000 \\
\hline
\end{tabular}
The routing perfomance drops by 35\% when it is performed by
a CPU not directly connected to the PCI-Express link with the NIC.

\begin{lstlisting}[language=TeX]
 L3MISS: L3 cache misses
 L2MISS: L2 cache misses (including other core's L2 cache *hits*)
 L3HIT : L3 cache hit ratio (0.00-1.00)
 L2HIT : L2 cache hit ratio (0.00-1.00)
 L3CLK : ratio of CPU cycles lost due to L3 cache misses (0.00-1.00), in some cases could be >1.0 due to a higher memory latency
 L2CLK : ratio of CPU cycles lost due to missing L2 cache but still hitting  L3 cache (0.00-1.00)
 L3OCC : L3 occupancy (in KBytes)


 Core (SKT) | L3MISS | L2MISS | L3HIT | L2HIT | L3CLK | L2CLK |  L3OCC

   0    0     6209       24 K    0.75    0.32    0.19    0.12      480
   1    0       13      546      0.98    0.12    0.01    0.14      640
   2    0      154     4746      0.97    0.16    0.03    0.18       40
   3    0       16     1452      0.99    0.14    0.01    0.15        0
   4    0       68     1164      0.94    0.25    0.00    0.00       80
   5    0       10      537      0.98    0.13    0.01    0.14      160
   6    0        5      538      0.99    0.13    0.01    0.14        0
   7    0       10      543      0.98    0.12    0.01    0.13        0
   8    0       10      548      0.98    0.11    0.01    0.11        0
   9    0     2002 K   2798 K    0.28    0.74    0.20    0.02     2360
  10    1      790       12 K    0.94    0.14    0.08    0.24     4640
  11    1       29      985      0.97    0.12    0.02    0.15        0
  12    1       22      569      0.96    0.11    0.02    0.09        0
  13    1      325     2112      0.85    0.15    0.05    0.06       40
  14    1      238     1328      0.82    0.16    0.05    0.05      120
  15    1       28      526      0.95    0.16    0.02    0.07       40
  16    1       32      572      0.94    0.13    0.02    0.07        0
  17    1       27      563      0.95    0.11    0.02    0.08        0
  18    1       20      563      0.96    0.12    0.01    0.09       40
  19    1       30      606      0.95    0.11    0.02    0.09       40
  20    0      158     1017      0.84    0.07    0.08    0.13       40
  21    0       10      553      0.98    0.13    0.02    0.20        0
  22    0       10      739      0.99    0.10    0.01    0.16       40
  23    0        7      574      0.99    0.13    0.01    0.18        0
  24    0       10      578      0.98    0.14    0.01    0.16        0
  25    0        4      535      0.99    0.14    0.01    0.16        0
  26    0       11      551      0.98    0.12    0.01    0.16        0
  27    0       11      550      0.98    0.13    0.01    0.13        0
  28    0       10      560      0.98    0.13    0.01    0.14        0
  29    0       97     5729      0.98    0.07    0.00    0.05        0
  30    1      185      911      0.80    0.08    0.10    0.12        0
  31    1       19      572      0.97    0.11    0.02    0.13        0
  32    1       15      556      0.97    0.13    0.02    0.13       80
  33    1       23      591      0.96    0.13    0.02    0.12       40
  34    1       21      555      0.96    0.12    0.01    0.09        0
  35    1       21      539      0.96    0.15    0.02    0.09      120
  36    1       23      555      0.96    0.13    0.02    0.09        0
  37    1       63      684      0.91    0.18    0.01    0.02        0
  38    1       16      555      0.97    0.14    0.01    0.10        0
  39    1     4826     7108      0.32    0.52    0.24    0.02        0
------------------------------------------------------------------------
 SKT    0     2009 K   2844 K    0.29    0.73    0.20    0.02     3840
 SKT    1     6753       32 K    0.79    0.26    0.10    0.08     5160
------------------------------------------------------------------------
 TOTAL  *     2015 K   2877 K    0.30    0.73    0.19    0.02      N/A
\end{lstlisting}

CPU~9 performs the actual forwarding, while CPU~10
is busy with the QPI communication overhead.

\begin{lstlisting}
Intel(r) QPI data traffic estimation in bytes (data traffic coming to CPU/socket through QPI links):

               QPI0     QPI1    |  QPI0   QPI1  
----------------------------------------------------------------------------------------------
 SKT    0       90 M     90 M   |    0%     0%   
 SKT    1       70 M     71 M   |    0%     0%   
----------------------------------------------------------------------------------------------
Total QPI incoming data traffic:  323 M     QPI data traffic/Memory controller traffic: 0.39
\end{lstlisting}
Both QPI links are used.


perf top -C 9
\begin{lstlisting}
  21.25%  [kernel]  [k] _raw_spin_lock
  11.61%  [kernel]  [k] memcpy
   6.69%  [kernel]  [k] fib_table_lookup
   6.60%  [kernel]  [k] skb_gro_reset_offset
   4.55%  [kernel]  [k] udp_gro_receive
   3.96%  [kernel]  [k] mlx4_en_xmit
   3.44%  [kernel]  [k] mlx4_en_process_rx_cq
   2.93%  [kernel]  [k] mlx4_en_poll_tx_cq
\end{lstlisting}

perf top -C 10
\begin{lstlisting}
   9.77%  [kernel]  [k] find_busiest_group
   3.47%  [kernel]  [k] cpumask_next_and
   3.01%  [kernel]  [k] _raw_spin_lock
   2.93%  [kernel]  [k] ktime_get
   2.59%  [kernel]  [k] mlx4_en_DUMP_ETH_STATS
   2.58%  [kernel]  [k] run_timer_softirq
   2.36%  [kernel]  [k] idle_cpu
   2.22%  [kernel]  [k] __schedule
\end{lstlisting}

	
	%=========================================================================
% (c) 2014, 2015 Josef Lusticky

% FINAL
\subsection{Measurement 6 - 32 IPv4 flows with irqbalance daemon}
The measurement includes the {\it{irqbalance}} daemon enabled.
The {\it{irqbalance}} daemon is responsible for dynamically assigning the interrupts to CPUs
using the files found under /proc/irq/{\it{NUMBER}}/smp\_affinity,
as described in subsection~\ref{sub:analysis-settings-procfs}.
\\
\\
\begin{tabular}{ | l | l | l | l | }
\hline
Frame size & \% of link & bandwidth & frame rate \\
\hline
64     &  8.99\% &  3.60~Gb/s & 5~350~000 \\
594    & 68.15\% & 27.26~Gb/s & 5~550~000 \\
1518   & 98.50\% & 39.40~Gb/s & 3~202~210 \\
AMS-IX & 88.25\% & 35.30~Gb/s & 5~800~000 \\
\hline
\end{tabular}
\\
\\
The scaling mechanisms of the Linux kernel take advantage of interrupt assignment
done by the {\it{irqbalance}} daemon.
The server is able to route almost 36~Gbps of the simulated AMS-IX internet traffic.
The measurement further confirms that the scaling mechanisms are sensitive to the frame size.
The measurement featuring 1518~octet frames was configured to use 98.5\% of the link bandwidth.
\\
The following listing shows that the {\it{irqbalance}} daemon assigned IRQs to CPUs 11-18, 30 and 33-39.
The CPUs 12-19 are serving RX interrupts, while the CPUs 30-39 are serving TX interrupts
({\it{en29d1}} represents the receiving interface).

\newpage

\begin{landscape}
\vspace*{\fill}
\begin{lstlisting}
     CPU12 CPU13 CPU14 CPU15 CPU16 CPU17 CPU18 CPU19  CPU30  CPU33  CPU34  CPU35  CPU36  CPU37  CPU38  CPU39
178:     0     0     0     0     0     0     0     0      0 292448      0      0      0      0      0      0 en29-0
179:     0     0     0     0     0     0     0     0      0      0 292978      0      0      0      0      0 en29-1
180:     0     0     0     0     0     0     0     0      0      0      0 292698      0      0      0      0 en29-2
181:     0     0     0     0     0     0     0     0      0      0      0      0 286435      0      0      0 en29-3
182:     0     0     0     0     0     0     0     0      0      0      0      0      0 282449      0      0 en29-4
183:     0     0     0     0     0     0     0     0      0      0      0      0      0      0 288839      0 en29-5
184:     0     0     0     0     0     0     0     0      0      0      0      0      0      0      0 327901 en29-6
185:     0     0     0     0     0     0     0     0 325935      0      0      0      0      0      0      0 en29-7
186: 53145     0     0     0     0     0     0     0      0      0      0      0      0      0      0      0 en29d1-0
187:     0 53090     0     0     0     0     0     0      0      0      0      0      0      0      0      0 en29d1-1
188:     0     0 39978     0     0     0     0     0      0      0      0      0      0      0      0      0 en29d1-2
189:     0     0     0 40484     0     0     0     0      0      0      0      0      0      0      0      0 en29d1-3
190:     0     0     0     0 40072     0     0     0      0      0      0      0      0      0      0      0 en29d1-4
191:     0     0     0     0     0 39982     0     0      0      0      0      0      0      0      0      0 en29d1-5
192:     0     0     0     0     0     0 40488     0      0      0      0      0      0      0      0      0 en29d1-6
193:     0     0     0     0     0     0     0 43262      0      0      0      0      0      0      0      0 en29d1-6
\end{lstlisting}
\vspace*{\fill}
\end{landscape}

\noindent
\\
The listing bellow shows the cache use.
\begin{lstlisting}[language=TeX]
 L3MISS: L3 cache misses 
 L2MISS: L2 cache misses (including other core's L2 cache *hits*) 
 L3HIT : L3 cache hit ratio (0.00-1.00)
 L2HIT : L2 cache hit ratio (0.00-1.00)
 L3CLK : ratio of CPU cycles lost due to L3 cache misses (0.00-1.00), in some cases could be >1.0 due to a higher memory latency
 L2CLK : ratio of CPU cycles lost due to missing L2 cache but still hitting L3 cache (0.00-1.00)

 Core (SKT) | L3MISS | L2MISS | L3HIT | L2HIT | L3CLK | L2CLK |  L3OCC

   0    0     1076       11 K    0.91    0.17    0.01    0.01      120
   1    0      574     4139      0.86    0.12    0.10    0.14       40
   2    0      241     1421      0.83    0.18    0.12    0.12        0
   3    0      658       11 K    0.94    0.09    0.07    0.24        0
   4    0       19      580      0.97    0.14    0.02    0.12        0
   5    0       78      363      0.79    0.25    0.03    0.03        0
   6    0       65      710      0.91    0.15    0.04    0.10       40
   7    0       19      544      0.97    0.12    0.02    0.10       40
   8    0       13      648      0.98    0.11    0.01    0.08        0
   9    0       32      582      0.95    0.11    0.02    0.08        0
  10    1     1007     3598      0.72    0.06    0.72    0.37       40
  11    1     4452     3802 K    1.00    0.45    0.00    0.10       40
  12    1     4932     3770 K    1.00    0.42    0.00    0.10      240
  13    1     6273     4131 K    1.00    0.51    0.00    0.09       40
  14    1     5086     4211 K    1.00    0.52    0.00    0.09       80
  15    1     4762     4211 K    1.00    0.50    0.00    0.10       80
  16    1     4453     4111 K    1.00    0.54    0.00    0.09       80
  17    1     4680     4124 K    1.00    0.56    0.00    0.09      160
  18    1     4737     4275 K    1.00    0.48    0.00    0.10       80
  19    1      105      573      0.82    0.18    0.18    0.17        0
  20    0      170      979      0.83    0.06    0.07    0.10       40
  21    0       16      692      0.98    0.10    0.01    0.12        0
  22    0       12      570      0.98    0.12    0.01    0.13        0
  23    0       15      839      0.98    0.10    0.01    0.13        0
  24    0       13      532      0.98    0.16    0.02    0.14        0
  25    0       22      441      0.95    0.21    0.01    0.06        0
  26    0       14      540      0.97    0.14    0.02    0.13        0
  27    0        9      516      0.98    0.15    0.01    0.13        0
  28    0      268     2659      0.90    0.09    0.10    0.18       40
  29    0      239      916      0.74    0.15    0.10    0.06        0
  30    1     1238     2816 K    1.00    0.60    0.00    0.27      680
  31    1       54      560      0.90    0.25    0.09    0.19       80
  32    1     3416     9666      0.65    0.20    0.72    0.27       40
  33    1       14 K   3887 K    1.00    0.64    0.00    0.27      640
  34    1       16 K   3969 K    1.00    0.63    0.00    0.27      960
  35    1       15 K   3881 K    1.00    0.64    0.00    0.27      960
  36    1       14 K   3878 K    1.00    0.63    0.00    0.27      880
  37    1       14 K   3920 K    1.00    0.63    0.00    0.27      440
  38    1       17 K   4014 K    1.00    0.63    0.00    0.27      600
  39    1     2106     2853 K    1.00    0.60    0.00    0.28     1240
------------------------------------------------------------------------
 SKT    0     3553       41 K    0.91    0.13    0.01    0.03      320
 SKT    1      141 K     61 M    1.00    0.57    0.00    0.14     7360
------------------------------------------------------------------------
 TOTAL  *      144 K     61 M    1.00    0.57    0.00    0.14      N/A 
\end{lstlisting}
The measurement featuring 1518~octet frames is the first measurement saturating the 40~Gbps Ethernet connection.
Intel PCM can be used to monitor the PCI-Express utilisation:
\begin{lstlisting}
Skt | PCIe Rd (B) | PCIe Wr (B)
 0         5270 K          86 K
 1           11 G        5422 M
--------------------------------
 *           11 G        5422 M
\end{lstlisting}
The PCI-Express link could be saturated when forwarding bidirectional traffic
- the PCI-Express 3.0 x8 throughput is 7~876.8~MB/s as calculated in section~\ref{sec:40gbe-throughput}.
Note, that there seems to be a bug in Intel PCM related to displaying
the PCIe Read bandwidth - it always shows double the expected value (11~Gigabytes does not make sense).

	
	%=========================================================================
% (c) 2014, 2015 Josef Lusticky

% FINAL
\subsection{Measurement 7 - 32 IPv4 flows with manual IRQ affinity mappings}
While {\it{irqbalance}} mapped the interrupts intelligently, it is always worth checking the mappings.
The dynamic mappings made by the {\it{irqbalance}} daemon can change during the run-time, which may lead
to unpredictable performance drops.
\\
The following listing shows the interrupt mapping scheme used during this measurement.
Unlike the mapping assigned by the {\it{irqbalance}} daemon,
this mapping targets both RX and TX interrupts to 8 CPUs only.
Additionally, Transmission Packet Steering (XPS) mechanism was configured
to maps each exclusively to a single CPUs, as described in section~\ref{sec:linux-scaling}.
\begin{lstlisting}[language=TeX]
echo 1 > /proc/irq/default_smp_affinity     # mask for new registered irqs
echo 0 | tee /proc/irq/*/smp_affinity_list  # assign all IRQs to CPU 0

echo 18 > /proc/irq/177/smp_affinity_list   # assign mlx4-async IRQ to CPU 18

echo 10 > /proc/irq/178/smp_affinity_list   # enp129s0-0 IRQ to CPU 10
echo 11 > /proc/irq/179/smp_affinity_list
echo 12 > /proc/irq/180/smp_affinity_list
echo 13 > /proc/irq/181/smp_affinity_list
echo 14 > /proc/irq/182/smp_affinity_list
echo 15 > /proc/irq/183/smp_affinity_list
echo 16 > /proc/irq/184/smp_affinity_list
echo 17 > /proc/irq/185/smp_affinity_list   # enp192s0-7 IRQ to CPU 17

echo 10 > /proc/irq/186/smp_affinity_list   # enp192s0d1-0 IRQ to CPU 10
echo 11 > /proc/irq/187/smp_affinity_list
echo 12 > /proc/irq/188/smp_affinity_list
echo 13 > /proc/irq/189/smp_affinity_list
echo 14 > /proc/irq/190/smp_affinity_list
echo 15 > /proc/irq/191/smp_affinity_list
echo 16 > /proc/irq/192/smp_affinity_list
echo 17 > /proc/irq/193/smp_affinity_list   # enp192s0d1-7 IRQ to CPU 17

# clear XPS on both interfaces
echo "0" | tee /sys/class/net/enp192s0/queues/tx-*/xps_cpus
echo "0" | tee /sys/class/net/enp192s0d1/queues/tx-*/xps_cpus

# use the IRQ mask to assign XPS
cat /proc/irq/178/smp_affinity > /sys/class/net/enp129s0/queues/tx-0/xps_cpus
cat /proc/irq/179/smp_affinity > /sys/class/net/enp129s0/queues/tx-1/xps_cpus
cat /proc/irq/180/smp_affinity > /sys/class/net/enp129s0/queues/tx-2/xps_cpus
cat /proc/irq/181/smp_affinity > /sys/class/net/enp129s0/queues/tx-3/xps_cpus
cat /proc/irq/182/smp_affinity > /sys/class/net/enp129s0/queues/tx-4/xps_cpus
cat /proc/irq/183/smp_affinity > /sys/class/net/enp129s0/queues/tx-5/xps_cpus
cat /proc/irq/184/smp_affinity > /sys/class/net/enp129s0/queues/tx-6/xps_cpus
cat /proc/irq/185/smp_affinity > /sys/class/net/enp129s0/queues/tx-7/xps_cpus

cat /proc/irq/186/smp_affinity > /sys/class/net/enp129s0d1/queues/tx-0/xps_cpus
cat /proc/irq/187/smp_affinity > /sys/class/net/enp129s0d1/queues/tx-1/xps_cpus
cat /proc/irq/188/smp_affinity > /sys/class/net/enp129s0d1/queues/tx-2/xps_cpus
cat /proc/irq/189/smp_affinity > /sys/class/net/enp129s0d1/queues/tx-3/xps_cpus
cat /proc/irq/190/smp_affinity > /sys/class/net/enp129s0d1/queues/tx-4/xps_cpus
cat /proc/irq/191/smp_affinity > /sys/class/net/enp129s0d1/queues/tx-5/xps_cpus
cat /proc/irq/192/smp_affinity > /sys/class/net/enp129s0d1/queues/tx-6/xps_cpus
cat /proc/irq/193/smp_affinity > /sys/class/net/enp129s0d1/queues/tx-7/xps_cpus
\end{lstlisting}
The {\it{mlx4}} driver uses combined interrupts for RX and TX,
therefore each mapped CPU serves RX and TX interrupts for the same packets.
Such mapping should lead to a better cache utilisation than in the previous measurement.
\\
\\
\begin{tabular}{ | l | l | l | l |}
\hline
Frame size & \% of link & bandwidth & frame rate \\
\hline
64     &  8.99\% &  3.60~Gb/s & 5~350~000 \\
594    & 68.15\% & 27.26~Gb/s & 5~550~000 \\
1518   & 99.60\% & 39.40~Gb/s & 3~202~210 \\
AMS-IX & 88.25\% & 35.30~Gb/s & 5~800~000 \\
\hline
\end{tabular}
\\
\\
The throughput performance with manual IRQ mappings is equal to the mappings set by the {\it{irqbalance}} daemon.
The measurement was also configured to use 128, 256, 512, 768, 1024 and 1280~byte sized frames
and the following graph presents the results.
\begin{figure}[H]
	\centering
	\includegraphics[width=12cm,keepaspectratio]{fig/frames.png}
\end{figure}
\noindent
The system is able to perform forwarding at nearly line rate speed with frames 1024~B and larger.
The following listing shows the actual interrupt mappings obtained from the /proc/interrupts file.
\newpage
%177:       0          0          0          0          0          0          0          0       1655  mlx4-async
\begin{landscape}
\vspace*{\fill}
\begin{lstlisting}
       CPU10      CPU11      CPU12      CPU13      CPU14      CPU15      CPU16      CPU17      CPU18
178:  298717          0          0          0          0          0          0          0          0  enp129s0-0
179:       0     294264          0          0          0          0          0          0          0  enp129s0-1
180:       0          0     291502          0          0          0          0          0          0  enp129s0-2
181:       0          0          0     296514          0          0          0          0          0  enp129s0-3
182:       0          0          0          0     302588          0          0          0          0  enp129s0-4
183:       0          0          0          0          0     294017          0          0          0  enp129s0-5
184:       0          0          0          0          0          0     294314          0          0  enp129s0-6
185:       0          0          0          0          0          0          0     302088          0  enp129s0-7
186:  402863          0          0          0          0          0          0          0          0  enp129s0d1-0
187:       0     411668          0          0          0          0          0          0          0  enp129s0d1-1
188:       0          0     416164          0          0          0          0          0          0  enp129s0d1-2
189:       0          0          0     422023          0          0          0          0          0  enp129s0d1-3
190:       0          0          0          0     421805          0          0          0          0  enp129s0d1-4
191:       0          0          0          0          0     415759          0          0          0  enp129s0d1-5
192:       0          0          0          0          0          0     402918          0          0  enp129s0d1-6
193:       0          0          0          0          0          0          0     413299          0  enp129s0d1-7
\end{lstlisting}
\vspace*{\fill}
\end{landscape}

\noindent
The RX and TX interrupts are spread across 8 CPUs.
The advantage of this mapping against the mapping done by the {\it{irqbalance}} daemon
is that it requires half the CPUs and the IRQ serving should cause fewer cache misses as well.
The listing bellow shows the cache statistics.
\begin{lstlisting}[language=TeX]
 L3MISS: L3 cache misses 
 L2MISS: L2 cache misses (including other core's L2 cache *hits*) 
 L3HIT : L3 cache hit ratio (0.00-1.00)
 L2HIT : L2 cache hit ratio (0.00-1.00)
 L3CLK : ratio of CPU cycles lost due to L3 cache misses (0.00-1.00)
 L2CLK : ratio of CPU cycles lost due to missing L2 cache (0.00-1.00)
 L3OCC : L3 occupancy (in KBytes)
 
 Core (SKT) | L3MISS | L2MISS | L3HIT | L2HIT | L3CLK | L2CLK |  L3OCC
   0    0     5799       33 K    0.82    0.24    0.19    0.18        0
   1    0      122     2488      0.95    0.15    0.03    0.14        0
   2    0      374     3928      0.90    0.19    0.00    0.00        0
   3    0        8      532      0.98    0.15    0.01    0.08       80
   4    0        6      519      0.99    0.16    0.00    0.08        0
   5    0       13      540      0.98    0.13    0.01    0.07        0
   6    0       12      552      0.98    0.14    0.01    0.07        0
   7    0       12      560      0.98    0.14    0.01    0.08      120
   8    0        9      547      0.98    0.13    0.01    0.08        0
   9    0       25      593      0.96    0.13    0.02    0.08        0
  10    1     9527     6014 K    1.00    0.51    0.00    0.14     1080
  11    1     9633     6096 K    1.00    0.52    0.00    0.14     1280
  12    1     9440     6257 K    1.00    0.53    0.00    0.15     1280
  13    1     9349     6069 K    1.00    0.54    0.00    0.14      960
  14    1     9147     6007 K    1.00    0.50    0.00    0.14     1000
  15    1     9270     6043 K    1.00    0.52    0.00    0.15     1160
  16    1     9382     6230 K    1.00    0.52    0.00    0.15     1560
  17    1     9249     6055 K    1.00    0.56    0.00    0.14      920
  18    1     1395     8002      0.83    0.15    0.29    0.29       40
  19    1      198     1129      0.82    0.12    0.13    0.14        0
  20    0      171     1042      0.84    0.08    0.07    0.11       40
  21    0       35     1176      0.97    0.16    0.02    0.17        0
  22    0       24      715      0.97    0.13    0.01    0.11       40
  23    0       18      537      0.97    0.15    0.01    0.09        0
  24    0        5      536      0.99    0.16    0.00    0.09        0
  25    0       11      533      0.98    0.14    0.01    0.08        0
  26    0       15      541      0.97    0.14    0.01    0.08        0
  27    0       11      538      0.98    0.16    0.01    0.09        0
  28    0       14      547      0.97    0.15    0.01    0.09        0
  29    0       46      672      0.93    0.12    0.03    0.10        0
  30    1      557     1200      0.54    0.34    0.03    0.01        0
  31    1      181      612      0.70    0.21    0.26    0.14       40
  32    1      192      604      0.68    0.23    0.25    0.12        0
  33    1      246      634      0.61    0.22    0.31    0.12        0
  34    1      202      622      0.68    0.22    0.28    0.15        0
  35    1      144      599      0.76    0.25    0.20    0.15        0
  36    1      127      620      0.80    0.22    0.17    0.15        0
  37    1      142      619      0.77    0.20    0.21    0.16       40
  38    1      132      685      0.81    0.11    0.13    0.15        0
  39    1      308      921      0.22    0.43    0.38    0.02      360
------------------------------------------------------------------------
 SKT    0     6730       50 K    0.87    0.21    0.03    0.04      280
 SKT    1       85 K     48 M    1.00    0.52    0.00    0.14     9720
------------------------------------------------------------------------
 TOTAL  *       92 K     48 M    1.00    0.52    0.00    0.14     N/A
\end{lstlisting}
As expected, the total cache miss count is lower with the manual IRQ mappings.
\\
The manual mappings are the best IRQ affinity settings in terms of number of CPUs used,
cache use and predictability.

	
	%=========================================================================
% (c) 2014, 2015 Josef Lusticky

% FINAL
\subsection{Measurement 8 - single IPv6 flow}

echo 0 > /proc/sys/net/ipv6/conf/all/disable\_ipv6
echo 1 > /proc/sys/net/ipv6/conf/all/forwarding
again, forwarding on only selected interfaces can be enabled.
ip6tables -F
ip6tables -X
ip6tables -L

ip -6 addr add 1::1/64 dev enp6s0d1
ip -6 addr add 2::1/64 dev enp6s0


%Single IPv6 flow.
%To test the difference between IPv4 and IPv6 routing lookup performance.

\begin{tabular}{ | l | l | l | l | }
\hline
Frame size & \% of link & frame rate \\
\hline
78     &  1.76\% &  0.71~Gb/s & 9~000~000 \\
594    & 11.05\% &  4.42~Gb/s & 9~000~000 \\
1518   & 27.68\% & 11.07~Gb/s & 9~000~000 \\
AMS-IX & 13.69\% &  5.48~Gb/s & 9~000~000 \\
\hline
\end{tabular}

perf top -C 14
\begin{lstlisting}
  11.75%  [kernel]  [k] ip6t_do_table
  10.37%  [kernel]  [k] _raw_spin_lock
   8.43%  [kernel]  [k] fib6_lookup
   4.93%  [kernel]  [k] ip6_forward
   3.66%  [kernel]  [k] fib6_get_table
   3.48%  [kernel]  [k] ip6_rcv_finish
   3.40%  [kernel]  [k] build_skb
   3.26%  [kernel]  [k] __netif_receive_skb_core
   3.01%  [kernel]  [k] mlx4_en_complete_rx_desc
   3.00%  [kernel]  [k] _raw_read_unlock_bh
   2.65%  [kernel]  [k] mlx4_en_process_rx_cq
   2.62%  [kernel]  [k] dst_release
\end{lstlisting}



\section{Upstream mainline kernel 3.19.2}

	%=========================================================================
% (c) 2014, 2015 Josef Lusticky

% FINAL
\subsection{Measurement 9 - 32 independent IPv6 flows with irqbalance}

\begin{tabular}{ | l | l | l | l | }
\hline
Frame size & \% of link & bandwidth & frame rate \\
\hline
78     &  4.90\% &  1.96~Gb/s & 2~500~000 \\
594    & 31.32\% & 12.53~Gb/s & 2~550~000 \\
1518   & 98.00\% & 39.20~Gb/s & 3~185~955 \\
AMS-IX & 48.69\% & 19.48~Gb/s & 3~200~000 \\
\hline
\end{tabular}

The scaling mechanisms are sensitive to frame size.

Spin lock.

\begin{lstlisting}
  17.70%  [kernel]          [k] ip6_pol_route.isra.46
   8.18%  [kernel]          [k] _raw_spin_lock
   8.17%  [kernel]          [k] fib6_lookup
   6.43%  [kernel]          [k] _raw_read_lock_bh
   5.60%  [kernel]          [k] _raw_read_unlock_bh
   5.39%  [kernel]          [k] fib6_get_table
   4.79%  [kernel]          [k] dst_release
   3.37%  [kernel]          [k] ip6_finish_output2
   2.48%  [kernel]          [k] mlx4_en_xmit
   2.11%  [kernel]          [k] __netif_receive_skb_core
   2.08%  [kernel]          [k] memcpy
   2.04%  [kernel]          [k] mlx4_en_process_rx_cq
   1.32%  [kernel]          [k] mlx4_en_poll_tx_cq
   1.27%  [kernel]          [k] local_bh_enable_ip
   1.26%  [kernel]          [k] __pskb_pull_tail
   1.19%  [kernel]          [k] put_compound_page
\end{lstlisting}

	
	%=========================================================================
% (c) 2014, 2015 Josef Lusticky

% FINAL
\subsection{32 IPv4 flows on 8 queues}

\begin{tabular}{ | l | l | l | l | }
\hline
Frame size & \% of link & bandwidth & frame rate \\
\hline
64     &  5.29\% &  2.12~Gb/s & 3~150~000 \\ %DONE 2
594    & 43.59\% & 17.44~Gb/s & 3~550~000 \\ %DONE 2
1518   & 99.50\% & 39.80~Gb/s & 3~234~720 \\ %DONE 2
AMS-IX & 54.77\% & 21.91~Gb/s & 3~600~000 \\ %DONE 2
\hline
\end{tabular}


\begin{lstlisting}
 119:          0          0          0          0          0          0          0          0          0          0     393804          0          0          0          0          0          0          0          0          0          0          0          0          0          0          0          0          0          0          0          0          0          0          0          0          0          0          0          0          0  IR-PCI-MSI-edge      enp129s0-0
 120:          0          0          0          0          0          0          0          0          0          0          0     397112          0          0          0          0          0          0          0          0          0          0          0          0          0          0          0          0          0          0          0          0          0          0          0          0          0          0          0          0  IR-PCI-MSI-edge      enp129s0-1
 121:          0          0          0          0          0          0          0          0          0          0          0          0          0          0          0          0          0          0          0          0          0          0          0          0          0          0          0          0          0          0          0          0          0          0          0          0          0          0          0          0  IR-PCI-MSI-edge      enp129s0-2
 122:          0          0          0          0          0          0          0          0          0          0          0          0          0          0          0          0          0          0          0          0          0          0          0          0          0          0          0          0          0          0          0          0          0          0          0          0          0          0          0          0  IR-PCI-MSI-edge      enp129s0-3
 123:          0          0          0          0          0          0          0          0          0          0          0          0          0          0          0          0          0          0          0          0          0          0          0          0          0          0          0          0          0          0          0          0          0          0          0          0          0          0          0          0  IR-PCI-MSI-edge      enp129s0-4
 124:          0          0          0          0          0          0          0          0          0          0          0          0          0          0          0          3          0          0          0          0          0          0          0          0          0          0          0          0          0          0          0          0          0          0          0          0          0          0          0          0  IR-PCI-MSI-edge      enp129s0-5
 125:          0          0          0          0          0          0          0          0          0          0          0          0          0          0          0          0     268338          0          0          0          0          0          0          0          0          0          0          0          0          0          0          0          0          0          0          0          0          0          0          0  IR-PCI-MSI-edge      enp129s0-6
 126:          0          0          0          0          0          0          0          0          0          0          0          0          0          0          0          0          0     263689          0          0          0          0          0          0          0          0          0          0          0          0          0          0          0          0          0          0          0          0          0          0  IR-PCI-MSI-edge      enp129s0-7
 127:          0          0          0          0          0          0          0          0          0          0          8          0          0          0          0          0          0          0          0          0          0          0          0          0          0          0          0          0          0          0          0          0          0          0          0          0          0          0          0          0  IR-PCI-MSI-edge      eth1-0
 128:          0          0          0          0          0          0          0          0          0          0          0          3          0          0          0          0          0          0          0          0          0          0          0          0          0          0          0          0          0          0          0          0          0          0          0          0          0          0          0          0  IR-PCI-MSI-edge      eth1-1
 129:          0          0          0          0          0          0          0          0          0          0          0          0          0          0          0          0          0          0          0          0          0          0          0          0          0          0          0          0          0          0          0          0          0          0          0          0          0          0          0          0  IR-PCI-MSI-edge      eth1-2
 130:          0          0          0          0          0          0          0          0          0          0          0          0          0          0          0          0          0          0          0          0          0          0          0          0          0          0          0          0          0          0          0          0          0          0          0          0          0          0          0          0  IR-PCI-MSI-edge      eth1-3
 131:          0          0          0          0          0          0          0          0          0          0          0          0          0          0     157003          0          0          0          0          0          0          0          0          0          0          0          0          0          0          0          0          0          0          0          0          0          0          0          0          0  IR-PCI-MSI-edge      eth1-4
 132:          0          0          0          0          0          0          0          0          0          0          0          0          0          0          0     154495          0          0          0          0          0          0          0          0          0          0          0          0          0          0          0          0          0          0          0          0          0          0          0          0  IR-PCI-MSI-edge      eth1-5
 133:          0          0          0          0          0          0          0          0          0          0          0          0          0          0          0          0     118030          0          0          0          0          0          0          0          0          0          0          0          0          0          0          0          0          0          0          0          0          0          0          0  IR-PCI-MSI-edge      eth1-6
 134:          0          0          0          0          0          0          0          0          0          0          0          0          0          0          0          0          0     114088          0          0          0          0          0          0          0          0          0          0          0          0          0          0          0          0          0          0          0          0          0          0  IR-PCI-MSI-edge      eth1-7
\end{lstlisting}

Either the {\it{mlx4}} driver od the kernel itself has some serious problem with assigning IRQs.
This limits the forwarding performance significantly.


\section{Settings influence}

	%=========================================================================
% (c) 2014, 2015 Josef Lusticky

% FINAL
\subsection{Disabled Hyper-Threading}
Routing of a single IPv4 flow was tested with the Hyper-Threading technology disabled.
\\
\\
\begin{tabular}{ | l | l | l | l | }
\hline
Frame size & \% of link & bandwidth & frame rate \\
\hline
64     &  1.93\% &  0.77~Gb/s & 1~150~000 \\
594    & 14.12\% &  5.65~Gb/s & 1~150~000 \\
1518   & 35.37\% & 14.15~Gb/s & 1~150~000 \\
AMS-IX & 17.50\% &  7.00~Gb/s & 1~150~000 \\
\hline
\end{tabular}
\\
\\
A single core routing performance increased by 15\% with disabled Hyper-Threading.
\\
Routing of 32 IPv4 flows with manual IRQ affinity mappings was tested to investiage
how is the routing performance influenced when the networking code runs
on multiple cores with Hyper-Threading disabled.
\\
\\
\begin{tabular}{ | l | l | l | l | }
\hline
Frame size & \% of link & bandwidth & frame rate \\
\hline
64     &  9.07\% &  3.63~Gb/s & 5~400~000 \\
594    & 68.77\% & 27.51~Gb/s & 5~600~000 \\
1518   & 98.50\% & 39.40~Gb/s & 3~202~210 \\
AMS-IX & 89.77\% & 35.91~Gb/s & 5~900~000 \\
\hline
\end{tabular}
\\
\\
Disabled Hyper-Threading provides only 2\% performance increase when routing the AMS-IX traffic on multiple cores.
The logical cores which are not utilised do not decrease performance significantly.
Hyper-Threading is highly optimised on Intel Xeon E5-2660 v3.

	
The following measurements take only the AMS-IX distribution and 32 IPv4 flows into account.
The measurements are performed with HT disabled,
therefore the previous result of 5~900~000 frames per second can be used for comparision.

	%=========================================================================
% (c) 2014, 2015 Josef Lusticky

%FINAL
\subsection{Netdev\_budget}
The measurement features increased {\it{netdev\_budget}} NAPI parameter from its default value of 300 to 3~000.
The parameter is described in section~\ref{sec:linux-ingress}.
The {\it{netdev\_budget}} can be configured using {\it{procfs}}.
\begin{lstlisting}
echo 3000 > /proc/sys/net/core/netdev_budget
\end{lstlisting}
The following table shows the results.
\\
\\
\begin{tabular}{ | l | l | l | l | }
\hline
Frame size & \% of link & bandwidth & frame rate \\
\hline
AMS-IX & 88.25\% & 35.30~Gb/s & 5~800~000 \\
\hline
\end{tabular}
\\
\\
Increasing the {\it{netdev\_budget}} has no influence on routing performance,
which means that the raise softirq mechanism is highly optimised.


	%=========================================================================
% (c) 2014, 2015 Josef Lusticky

%FINAL
\subsection{Reverse path filter}
The measurement run with Reverse path filter enabled, which can be done using {\it{procfs}}.
The rp\_filter is described in section~\ref{sec:analysis-settings}.
\begin{lstlisting}[language=TeX]
echo 1 | tee /proc/sys/net/ipv4/conf/*/rp_filter
\end{lstlisting}


\begin{tabular}{ | l | l | l | l | }
\hline
Frame size & \% of link & bandwidth & frame rate \\
\hline
AMS-IX & 55.54\% & 22.21~Gb/s & 3~650~000 \\
\hline
\end{tabular}
\\
\\
The rp\_filter introduces a significant performance drop of about 37\%.
The following listing shows the ouput of the {\it{perf}} utility.
\begin{lstlisting}
perf top -C 10
  39.88%  [kernel]  [k] fib_table_lookup
   8.35%  [kernel]  [k] check_leaf.isra.7
   6.43%  [kernel]  [k] _raw_spin_lock
   2.94%  [kernel]  [k] mlx4_en_xmit
   2.40%  [kernel]  [k] memcpy
   2.32%  [kernel]  [k] __netif_receive_skb_core
   2.18%  [kernel]  [k] mlx4_en_process_rx_cq
   2.14%  [kernel]  [k] fib_validate_source
   1.40%  [kernel]  [k] ip_route_input_noref
\end{lstlisting}
The {\it{fib\_table\_lookup}} is performed twice for each packet
and it is therefore taking much more of the CPU time.
There is also new {\it{fib\_validate\_source}} function, which is calls
the actual {\it{fib\_table\_lookup}}.

In bidirectional routing with Reverse path filter enbaled,
it may be worth changing the default RSS hash key to a symetric one.
The RSS hash key is a taken as input by the Toeplitz hash function, as described in section~\ref{sec:linux-scaling}.
A symetric RSS key would lead to processing both directions of the same flow on the same CPU,
therefore the result of the {\it{fib\_table\_lookup}} can be obtained from the CPU's cache~\cite{symetric-rss}.


	%=========================================================================
% (c) 2014, 2015 Josef Lusticky

%FINAL
\subsection{SELinux}
Change to enforcing in the {\it{/etc/sysconfig/selinux}} file.

\begin{tabular}{ | l | l | l | l | }
\hline
Frame size & \% of link & bandwidth & frame rate \\
\hline
AMS-IX & 80.64\% & 32.26~Gb/s & 5~300~000 \\
\hline
\end{tabular}


	%netfilter
	%netfilter 1000 ip table rules
	%netfilter - masquerade == nat

\section{BGP routes}

	%=========================================================================
% (c) 2014, 2015 Josef Lusticky

The following measurement is perfomed with the Internet routes obtained from RIPE's BGP data.
The data contains 538~738 routes at the time of writing.
The complete setup of loading the routes into the kernel's FIB is described in appendix~\ref{app:bgp}.
The FIB statistics are exported via the {\it{/proc/net/fib\_triestat}} file, as described in
subsection~\ref{sub:analysis-metodology-collection}.
The following listing shows the content of the file after loading the Internet routes to the kernel's FIB.
\begin{lstlisting}
Basic info: size of leaf: 40 bytes, size of tnode: 40 bytes.
Main:
	Aver depth:     2.43
	Max depth:      8
	Leaves:         503308
	Prefixes:       538739
	Internal nodes: 114429
	  1: 58727  2: 26171  3: 14805  4: 7315  5: 4240  6: 2103  7: 1065  8: 2  17: 1
	Pointers: 995794
Null ptrs: 378058
Total size: 61373  kB

Counters:
---------
gets = 14129134
backtracks = 1913823
semantic match passed = 15973885
semantic match miss = 0
null node hit= 1360956
skipped node resize = 0

Local:
	Aver depth:     3.08
	Max depth:      4
	Leaves:         12
	Prefixes:       13
	Internal nodes: 7
	  1: 6  3: 1
	Pointers: 20
Null ptrs: 2
Total size: 2  kB

Counters:
---------
gets = 13959019
backtracks = 15201204
semantic match passed = 103193
semantic match miss = 0
null node hit= 8238092
skipped node resize = 0
\end{lstlisting}

\begin{tabular}{ | l | l | l | l | }
\hline
Frame size & \% of link & bandwidth & frame rate \\
\hline
AMS-IX & 77.60\% &  30.95~Gb/s & 5~100~000 \\
\hline
\end{tabular}

\begin{lstlisting}
 L3MISS: L3 cache misses 
 L2MISS: L2 cache misses (including other core's L2 cache *hits*) 
 L3HIT : L3 cache hit ratio (0.00-1.00)
 L2HIT : L2 cache hit ratio (0.00-1.00)
 L3CLK : ratio of CPU cycles lost due to L3 cache misses (0.00-1.00)
 L2CLK : ratio of CPU cycles lost due to missing L2 cache (0.00-1.00)
 L3OCC : L3 occupancy (in KBytes)

 Core (SKT) | L3MISS | L2MISS | L3HIT | L2HIT | L3CLK | L2CLK |  L3OCC

   0    0     4048       16 K    0.75    0.40    0.18    0.11      360
   1    0      574     3257      0.82    0.17    0.00    0.00      160
   2    0      515     4977      0.90    0.13    0.06    0.10       40
   3    0        5      539      0.99    0.11    0.00    0.12       40
   4    0        5      526      0.99    0.13    0.00    0.11        0
   5    0        5      532      0.99    0.12    0.00    0.10        0
   6    0        4      548      0.99    0.11    0.00    0.09        0
   7    0        4      541      0.99    0.12    0.00    0.09        0
   8    0        6      548      0.99    0.11    0.00    0.08       40
   9    0       21      550      0.96    0.13    0.01    0.07       40
  10    1     1410 K   9962 K    0.86    0.47    0.09    0.14     2480
  11    1     1400 K   9901 K    0.86    0.46    0.09    0.14     2160
  12    1     1396 K   9872 K    0.86    0.46    0.09    0.14     2200
  13    1     1396 K   9897 K    0.86    0.47    0.09    0.14     2000
  14    1     1402 K   9872 K    0.86    0.46    0.09    0.14     2720
  15    1     1395 K   9861 K    0.86    0.49    0.09    0.14     2160
  16    1     1409 K   9827 K    0.86    0.48    0.09    0.14     2200
  17    1     1402 K   9867 K    0.86    0.46    0.09    0.14     2240
  18    1      398      827      0.52    0.07    0.43    0.11        0
  19    1      200      891      0.78    0.07    0.08    0.09        0
  20    0      213      675      0.68    0.09    0.11    0.06        0
  21    0        7      656      0.99    0.09    0.01    0.13        0
  22    0        6      747      0.99    0.07    0.00    0.12       40
  23    0       13      548      0.98    0.12    0.01    0.12        0
  24    0        4      543      0.99    0.14    0.00    0.11       40
  25    0        5      529      0.99    0.13    0.00    0.10       80
  26    0       25      761      0.97    0.09    0.02    0.13        0
  27    0        5      707      0.99    0.09    0.00    0.12        0
  28    0        5      716      0.99    0.09    0.00    0.11        0
  29    0       45      643      0.93    0.11    0.03    0.09        0
  30    1      362      704      0.49    0.22    0.33    0.06       40
  31    1      206      643      0.68    0.21    0.24    0.12        0
  32    1      187      644      0.71    0.18    0.22    0.12        0
  33    1      188      635      0.70    0.38    0.24    0.13        0
  34    1      193      619      0.69    0.20    0.27    0.14        0
  35    1      208      614      0.66    0.11    0.34    0.15       40
  36    1      184      639      0.71    0.21    0.26    0.15        0
  37    1      194      642      0.70    0.19    0.26    0.13        0
  38    1      181      616      0.71    0.09    0.23    0.13        0
  39    1       22 K     29 K    0.24    0.16    0.85    0.06      200
------------------------------------------------------------------------
 SKT    0     5515       35 K    0.84    0.28    0.02    0.03       840
 SKT    1       11 M     79 M    0.86    0.47    0.09    0.14     18440
------------------------------------------------------------------------
 TOTAL  *       11 M     79 M    0.86    0.47    0.09    0.14     N/A
\end{lstlisting}



%=========================================================================
% (c) 2014, 2015 Josef Lusticky

\chapter{Conclusion}\label{chap:conclusion}
The thesis provides description of the principles behind packet processing in the Linux kernel, as well as
a general description of the 40 Gigabit Ethernet protocol and its performance limitation.
An advanced knowledge of the principles described in the thesis is required to
perform the correct system settings for maximum routing performance with the GNU/Linux operating system.

The CentOS~7 operating system was installed to perform the measurements.
The system features Linux kernel based on version 3.10.
Additionally, upstream kernel 3.19.2 was installed.
The Linux kernel routing performance was measured under different scenarios, including
simulation of a real internet traffic, based on the data provided by the Amsterdam Internet Exchange.
Custom frame-size distribution was configured for this purpose and used in the experiments.
%TODO BGP
The overall routing performance of the Linux kernel is sufficient for routing 35.91~Gbps
of the simulated Internet traffic on Intel Xeon CPU.
The CPU features 8~physical cores and the HyperThreading technology (16~ logical cores).
The Mellanox ConnectX-3 EN adapter is able to scale the packet processing to 8~cores.%TODO
The Linux kernel is able to route 5.9~millions frames per second on 8~physical cores.

40Gbit software router with GNU/Linux may provide a reasonable alternative to proprietary hardware-based routers.
The main bottleneck of the default installation is the {\it{irqbalance}} daemon, which
does not set the optimal IRQ affinity for multiqueue network adapters.
Thus, the scaling mechanisms implemented in the Linux kernel are not used optimally.
The thesis describes how to set IRQ affinity manually for the purpose of maximum throughput.

The Linux kernel uses advantage of its scaling mechanisms to perform well in network processing,
however, a single-core packet processing must be improved to achieve better results.
The Linux kernel is able to route approx.~1~000~000 %TODO
frames per second utilising a single core.
This is also the limitation of a single network flow processing,
since the scaling mechanisms implemented in the Linux kernel and network adapters
are based on processing each flow by a different CPU.

The thesis can be further extended through a comparison against hardware-based routers.
The future work may compare latency, throughput or power consumption of both systems.
Additionally, various user-space frameworks focused on improving the GNU/Linux network performance may be evaluated,
such as Data Plane Development Kit.
Further work extending this thesis may include comparison of
the frameworks with the results presented in the thesis.

 % viz. obsah.tex

  % Pouzita literatura
  % ----------------------------------------------
\ifczech
  \bibliographystyle{czechiso}
\else 
  \bibliographystyle{plain}
%  \bibliographystyle{alpha}
\fi
  \begin{flushleft}
  \bibliography{literatura} % viz. literatura.bib
  \end{flushleft}
  \appendix
  
  %=========================================================================
% (c) 2014, 2015 Josef Lusticky

\chapter{Netmap installation}
Here comes how to setup netmap if we use it.


\chapter{Quagga installation}
Here comes how to setup Quagga if we use it.
 % viz. prilohy.tex
\end{document}
