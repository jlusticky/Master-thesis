%=========================================================================
% (c) 2014, 2015 Josef Lusticky

\section{Server configuration}
The general system configuration used by default in all measurements is described in this chapter.
It consists of disabling netfilter, disabling SELinux, changing the scaling governor, disabling rp\_filter and
configuring the IP addresses acording to the scheme described in section~\ref{sec:setup-hardware}.

%If disabling the netfilter is not appropriate, flush all chains iptables -F,
%delete all user-defined chains iptables -X (which can be checked again by iptables -L).
Netfilter was disabled completely:
\begin{lstlisting}[language=TeX]
systemctl stop firewalld
systemctl disable firewalld  # do not execute firewalld on boot
\end{lstlisting}
SELinux was disabled by changing the SELINUX variable to {\it{disabled}} in /etc/sysconfig/selinux
(default is {\it{enforcing}}). The system was rebooted to apply:
\begin{lstlisting}
vi /etc/sysconfig/selinux
	SELINUX=disabled
reboot
\end{lstlisting}
The scaling governor was changed for each CPU to {\it{performance}} (default is {\it{powersave}}):
\begin{lstlisting}
echo performance | tee /sys/devices/system/cpu/cpu[0-9]*/cpufreq/scaling_governor
\end{lstlisting}
The rp\_filter was disabled on all interfaces (it should be disabled at least on the forwarding interfaces):
\begin{lstlisting}[language=TeX]
echo 0 | tee /proc/sys/net/ipv4/conf/*/rp_filter
\end{lstlisting}
IPv6 was enabled on all interfaces (it must be enabled at least on the forwarding interfaces):
\begin{lstlisting}
echo 0 > /proc/sys/net/ipv6/conf/all/disable_ipv6
\end{lstlisting}
IPv4 forwarding was enabled:
\begin{lstlisting}
echo 1 > /proc/sys/net/ipv4/ip_forward
\end{lstlisting}
IPv6 forwarding was enabled on all interfaces (it must be enabled at least on the forwarding interfaces):
\begin{lstlisting}
echo 1 > /proc/sys/net/ipv6/conf/all/forwarding
\end{lstlisting}
IPv4 neighbours were set:
\begin{lstlisting}
ip neigh add 192.0.2.2 lladdr 00:10:94:00:00:01 dev enp6s0d1
ip neigh add 192.0.2.6 lladdr 00:10:94:00:00:02 dev enp6s0
\end{lstlisting}
IPv6 neighbours were set:
\begin{lstlisting}
ip -6 neigh add 2001:db8:1::2 lladdr 00:10:94:00:00:03 dev enp6s0d1
ip -6 neigh add 2001:db8:2::6 lladdr 00:10:94:00:00:04 dev enp6s0
\end{lstlisting}
IPv4 addresses were assigned:
\begin{lstlisting}
ip addr add 192.0.2.1/30 broadcast 192.0.2.3 dev enp6s0d1
ip addr add 192.0.2.5/30 broadcast 192.0.2.7 dev enp6s0
\end{lstlisting}
IPv6 addresses were assigned:
\begin{lstlisting}
ip -6 addr add 2001:db8:1::1/64 dev enp6s0d1
ip -6 addr add 2001:db8:2::5/64 dev enp6s0
\end{lstlisting}

The routing performance of the upstream Linux kernel version 3.19.2 was further measured.
The ELRepo was added to available repositories and the kernel was installed.
The instructions to add the ELRepo repository are provided by the elrepo.org site.\footnote{\url{http://elrepo.org/}}
Afterwards, the {\it{kernel-ml}} package was installed:
\begin{lstlisting}
yum --enablerepo=elrepo-kernel install kernel-ml
\end{lstlisting}
The kernel can be set as default in the bootloader configuration.
The following command prints all available kernels on the system:
\begin{lstlisting}[language=TeX]
grep "submenu\|^\menuentry" /boot/grub2/grub.cfg | cut -d "'" -f2
    CentOS Linux (3.19.2-1.el7.elrepo.x86_64) 7 (Core)
    CentOS Linux (3.10.0-123.20.1.el7.x86_64) 7 (Core)
    CentOS Linux, with Linux 0-rescue-f0d872ef69224325807afba7449b362b
\end{lstlisting}
The 3.19.2 kernel can be set as default by changing the configuration of grub2:
\begin{lstlisting}[language=TeX]
grub2-set-default "CentOS Linux (3.19.2-1.el7.elrepo.x86_64) 7 (Core)"
\end{lstlisting}

The routes announced in public BGP were imported to the kernel's FIB to perform additional measurements.
Appendix~\ref{app:bgp} describes the step-by-step instructions on how to obtain the BGP table and import it to the FIB.
Any additional system settings for a particular measurement are described next to the result.
