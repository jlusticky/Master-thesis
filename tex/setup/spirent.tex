%=========================================================================
% (c) 2014, 2015 Josef Lusticky

\section{Spirent configuration}\label{sec:setup-spirent}
Custom iMix, named AMS-IX, was configured according to the AMS-IX statistics described in
subsection~\ref{sub:analysis-metodology-generation}.
Figure~\ref{fig:setup-amsix-imix} shows the configuration window for custom iMix definition
from the Spirent TestCenter Application.
One device was configured on each interface and it was assigned IPv4 and IPv6 addresses, as described in section~\ref{sec:setup-hardware}.
The {\it{Respond to ping}} option was enabled to test the connection.

\begin{figure}
	\centering
	\includegraphics[width=14.5cm,keepaspectratio]{fig/amsix-imix.png}
	\caption{AMS-IX iMix}
	\label{fig:setup-amsix-imix}
\end{figure}

Two traffic patterns were configured for each IP version - one generates a single L4 flow and the other
generates 32 L4 flows.
The single flow traffic pattern uses IP addresses according the scheme described in section~\ref{sec:setup-hardware}
and UDP source and destination port 1024.
Each flow within the traffic pattern with 32 L4 flows uses different UDP source and destination ports
within range from 1024 up to 1055.
The source and destination ports are the same for each flow.
The scheme is used for both IPv4 and IPv6 traffic patterns.
The purpose of the scheme is to use Receive Side Scaling to distribute the traffic to all CPUs uniformly.

To test the routing performance of the Linux kernel with full BGP table, 
another traffic pattern was configured.
The pattern uses randomly generated IP destination address for each packet
to avoid having the previous lookup result cached.
All traffic patterns can be configured to generate fixed frame size or custom AMS-IX iMix.
