%=========================================================================
% (c) 2014, 2015 Josef Lusticky

\chapter{Populating kernel's FIB with BGP routes}\label{app:bgp}
RIPE Network Coordination Centre provides BGP data on its
website.\footnote{{\url{http://www.ripe.net/data-tools/stats/ris/ris-raw-data}}}
For the measurements in this thesis the data of RIPE NCC Amsterdam from 9th December 2014
The actual data can be obtained from the RIPE NCC's
site.\footnote{{\url{http://data.ris.ripe.net/rrc00/latest-bview.gz}}}

The BGP data is in MRT Routing Information Export Format, which is described in RFC~6396.
The data can be obtained from the Quagga routing daemon
using {\it{dump bgp routes}} command.
RIPE NCC provides bgpdump utility to parse this file to a human-readable text file.
To build bgpdump on CentOS 7 issue the following commands:
\begin{lstlisting}[language=TeX]
yum install bzip2-devel zlib-devel    # dependecies
http://www.ris.ripe.net/source/bgpdump/libbgpdump-1.4.99.13.tgz
tar xf libbgpdump-1.4.99.13.tgz
./configure
make
make example
make install
\end{lstlisting}
Note that make example is required prior to make install.
Afterwards, bgpdump command should be avaliable (installed to /usr/local/bin/bgpdump).
The Makefile also supports make rpm target to generate rpm package (a wrong date in specfile needs to be corrected).

The bgpdump utility can parse the downloaded latest-bview.gz file directly (even wihout extracting first).
The following command extracts individual destination networks from the BGP data file and writes them to {\it{destinations}} file.
\begin{lstlisting}[language=TeX]
bgpdump latest-bview.gz | grep 'PREFIX' | grep '\.' | sed 's/PREFIX: \(.*\)/\1/' | grep -v '0.0.0.0/0' | uniq > destinations
\end{lstlisting}
The numbers of routes corresponds to the number of prefixes announced on the Internet.
Note that the first destination entry is default route 0.0.0.0/0 and the file also contains subnets such as 10.0.0.0/8 or 192.168.0.0/16,
which you may want to remove from the file prior to the next step.
Basically, you want to remove all the local routes from the generated file.
The local routes can be displayed using the following command.
\begin{lstlisting}
ip route show table local | grep '^local'
\end{lstlisting}
If you are connected remotely, such as through SSH,
you must also remove routes to your local PC, which would break existing connection when misconfigured.

The {\it{destinations}} file can be used to insert the router to the kernel's FIB table by ip route add command.
To insert subnets from the {\it{destinations}} file to the kernel's FIB, the following C program was written.
The neighbours of the local router, which are used to forward the packets, are specified in the {\it{gateways}} array.
This array needs to be adjusted prior to executing.
The destination subnets are then inserted to kernel's FIB via these neighbours.
The program takes a single argument, path to the {\it{destinations}} file generated as described above.
Note the {\it{execlp()}} function call on line 57, which performs the route insertions.
\begin{lstlisting}[numbers=left]
#include <stdio.h>
#include <stdlib.h>
#include <unistd.h>
#include <errno.h>
#include <sys/wait.h>

#if !defined(ARRAY_SIZE)
    #define ARRAY_SIZE(x) (sizeof((x)) / sizeof((x)[0]))
#endif

/* Define your gateway IP addresses */
char *gateways[] = { "192.0.2.6" }; //, "193.160.39.1" }; // , etc. };

int main(int argc, char *argv[])
{
  char *gw;
  char *buf;
  size_t size = 64;
  int i = 0;

  if (argc != 2)
  {
    fprintf(stderr, "Usage: %s dstfile\n", argv[0]);
    return EXIT_FAILURE;
  }

  FILE *f = fopen(argv[1], "r");
  if (f == NULL)
  {
    fprintf(stderr, "%s %d\n", argv[1], __LINE__);
    return EXIT_FAILURE;
  }

  buf = malloc(64);
  if (buf == NULL)
  {
    fprintf(stderr, "%s %d\n", argv[0], __LINE__);
    fclose(f);
    return EXIT_FAILURE;
  }

  ssize_t len;
  while ((len = getline(&buf, &size, f)) > 5) // get destination subnets
  {
    buf[len-1] = '\0';
    gw = gateways[i % ARRAY_SIZE(gateways)];
    pid_t pid = fork();
    if (pid == -1)
    {
      fprintf(stderr, "%s %d\n", argv[0], __LINE__);
      fclose(f);
      free(buf);
      return EXIT_FAILURE;
    }
    else if (pid == 0) // child
    {
      printf("ip route add %s via %s\n", buf, gw);
      execlp("ip", "ip", "route", "add", buf, "via", gw, NULL);
    }
    else // parent
    {
      wait(NULL); // wait for child
    }
    i++; // move to next gw
  }
  fclose(f);
  free(buf);
  return EXIT_SUCCESS;
}
\end{lstlisting}

The scripts and programs shown above can be found on the CD attached to this thesis.

\chapter{Updating the Mellanox ConnectX-3 EN firmware}\label{app:firmware}
The firmware can be updated using the Mellanox Firmware Tools (MFT), which can be downloaded
from the Mellanox management tools site.\footnote{{\url{http://www.mellanox.com/page/management_tools}}}
Installation of Mellanox Firmware Tools is described in MTF User Manual available at
the Mellanox documentation site.\footnote{{\url{http://www.mellanox.com/related-docs/MFT/MFT_user_manual.pdf}}}

The installation consists of unpacking the downloaded MFT archive and running the {\it{install.sh}} script.
This will install kernel-mft and mft packages.
The kernel-mft package contains required low-level kernel drivers, whereas the mft package contains utilities which use these drivers.
To start the Mellanox Software Tools service, which also loads the kernel drivers, run {\it{mst start}}.
Afterwards, the utilities from the mft package can be used, including the mlxfwmanager for updating the firmware.
The mlxfwmanager displays various information about the adapter, including the firmware version.

The newest firmware for the Mellanox ConnectX-3 adapters can be downloaded
from the Mellanox firmware site.\footnote{{\url{http://www.mellanox.com/page/firmware_table_ConnectX3EN}}}
Unzip the firmware and use the {\it{mlxfwmanager -u}} command in the same directory where the firmware was unzipped.
This will update the firmware.
After the update is complete, reboot is required to take effect.

Listing~\ref{lst:setup-firmware-update} shows the output of the firmware update procedure using the {\it{mlxfwmanager -u}} command.
The current firmware version is 2.31.5050, whereas the newer firmware placed in the current working directory is of version 2.32.5100.
The mlxfwmanager\_pci utility can be used to query the adpater information without starting the mst service.

\newpage

\begin{lstlisting}[language=TeX,label={lst:setup-firmware-update},caption={Firmware update procedure}]
[root@server]# mlxfwmanager -u
Querying Mellanox devices firmware ...

Device #1:
----------

  Device Type:      ConnectX3
  Part Number:      MCX314A-BCB_Ax
  Description:      ConnectX-3 EN network interface card; 40GigE; dual-port QSFP; PCIe3.0 x8 8GT/s; RoHS R6
  PSID:             MT_1090110023
  PCI Device Name:  /dev/mst/mt4099_pci_cr0
  Port1 MAC:        f452145e6c70
  Port2 MAC:        f452145e6c71
  Versions:         Current        Available     
     FW             2.31.5050      2.32.5100     
     PXE            3.4.0225       3.4.0306      

  Status:           Update required

---------
Found 1 device(s) requiring firmware update...

Perform FW update? [y/N]: y
Device #1: Updating FW ... Done

Restart needed for updates to take effect.
\end{lstlisting}

\chapter{General steps for maximum routing performance}\label{app:general-steps}
\begin{itemize}
\item determine how many RX queues does the NIC support by "ethtool -{}-show-channels {\it{ifname}}" and use all of them using
"ethtool -{}-set-channels {\it{ifname}} rx {\it{num}}"
\item set the number of TX queues at least the same as RX queues by "ethtool -{}-set-channels {\it{ifname}} tx {\it{num}}"
\item determine how many physical cores has the CPU, which is directly connected to the PCI-Express link with the NIC
\item if the number of physical cores is less then the number of RX queues, enable Hyper-Threading
\item enable all offloads - check for disabled by "ethtool -{}-show-offload {\it{ifname}}"
\item disable SELinux in /etc/sysconfig/selinux (and reboot to apply)
\item disable netfilter completely or at least disable connection tracking
\item set {\it{performance}} scaling governor for all CPUs
\item disable rp\_filter
%TODO \item disable C-states using the kernel parameter intel\_idle.max\_cstate=0
\item disable irqbalance daemon and assign IRQs manually
\item if the NIC does not support RSS, configure RPS
%TODO \item disable XPS to avoid passing the TX packet processing to a different CPU
\end{itemize}
