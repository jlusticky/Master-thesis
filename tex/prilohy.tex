%=========================================================================
% (c) 2014, 2015 Josef Lusticky
\chapter{Populating kernel's FIB with BGP routes}
RIPE Network Coordination Centre provides BGP data on its
website.\footnote{{\url{http://www.ripe.net/data-tools/stats/ris/ris-raw-data}}}
For the measurements in this thesis the data of RIPE NCC Amsterdam from 9th December 2014
The actual data can be obtained from the RIPE NCC's
site.\footnote{{\url{http://data.ris.ripe.net/rrc00/latest-bview.gz}}}

The BGP data is in MRT Routing Information Export Format, which is described in RFC~6396.
The data can be obtained from the Quagga routing daemon
using {\it{dump bgp routes}} command.
RIPE NCC provides bgpdump utility to parse this file to a human-readable text file.
To build bgpdump on CentOS 7 issue the following commands:
\begin{lstlisting}[language=TeX]
yum install bzip2-devel zlib-devel    # dependecies
http://www.ris.ripe.net/source/bgpdump/libbgpdump-1.4.99.13.tgz
tar xf libbgpdump-1.4.99.13.tgz
./configure
make
make example
make install
\end{lstlisting}
Note that make example is required prior to make install.
Afterwards, bgpdump command should be avaliable (installed to /usr/local/bin/bgpdump).
The Makefile also supports make rpm target to generate rpm package (a wrong date in specfile needs to be corrected).

The bgpdump utility can parse the downloaded latest-bview.gz file directly (even wihout extracting first).
The following command extracts individual destination networks from the BGP data file and writes them to {\it{destinations}} file.
\begin{lstlisting}[language=TeX]
bgpdump latest-bview.gz | grep 'PREFIX' | grep '\.' | sed 's/PREFIX: \(.*\)/\1/' | uniq > destinations
\end{lstlisting}
The numbers of routes corresponds to the number of prefixes announced on the Internet.
Note that the first destination entry is default route 0.0.0.0/0 and the file also contains subnets such as 10.0.0.0/8 or 192.168.0.0/16,
you may want to remove these subnets from the file prior to the next step.

The {\it{destinations}} file can be used to insert the router to the kernel's FIB table by ip route add command.
To insert subnets from the {\it{destinations}} file to the kernel's FIB, the following C program was written.
The neighbours of the local router, which are used to forward the packets, are specified in the {\it{gateways}} array.
This array needs to be adjusted prior to executing.
The destination subnets are then inserted to kernel's FIB via these neighbours.
The program takes a single argument, path to the {\it{destinations}} file generated as described above.
Note the {\it{execlp()}} function call on line 57, which performs the route insertions.
\begin{lstlisting}[numbers=left]
#include <stdio.h>
#include <stdlib.h>
#include <unistd.h>
#include <errno.h>
#include <sys/wait.h>

#if !defined(ARRAY_SIZE)
    #define ARRAY_SIZE(x) (sizeof((x)) / sizeof((x)[0]))
#endif

char *gateways[] = { "192.168.56.1", "193.160.39.1" }; // , etc. };

int main(int argc, char *argv[])
{
  char *gw;
  char *buf;
  size_t size = 64;
  int i = 0;

  if (argc != 2)
  {
    fprintf(stderr, "Usage: %s dstfile\n", argv[0]);
    return EXIT_FAILURE;
  }

  FILE *f = fopen(argv[1], "r");
  if (f == NULL)
  {
    fprintf(stderr, "%s %d\n", argv[1], __LINE__);
    return EXIT_FAILURE;
  }

  buf = malloc(64);
  if (buf == NULL)
  {
    fprintf(stderr, "%s %d\n", argv[0], __LINE__);
    fclose(f);
    return EXIT_FAILURE;
  }

  ssize_t len;
  while ((len = getline(&buf, &size, f)) > 5) // get destination subnets
  {
    buf[len-1] = '\0';
    gw = gateways[i % ARRAY_SIZE(gateways)];
    pid_t pid = fork();
    if (pid == -1)
    {
      fprintf(stderr, "%s %d\n", argv[0], __LINE__);
      fclose(f);
      free(buf);
      return EXIT_FAILURE;
    }
    else if (pid == 0) // child
    {
      printf("ip route add %s via %s\n", buf, gw);
      execlp("ip", "ip", "route", "add", buf, "via", gw, NULL);
    }
    else // parent
    {
      wait(NULL); // wait for child
    }
    i++; // move to next gw
  }
  fclose(f);
  free(buf);
  return EXIT_SUCCESS;
}
\end{lstlisting}

The scripts and programs shown above can be found on the CD attached to this thesis.


\chapter{Steps to achieve maximum routing performance}
\begin{itemize}
\item enable Hyper-Threading
\item disable SELinux in /etc/sysconfig/selinux
\item disable C-states using the kernel parameter intel\_idle.max\_cstate=0
\item disable loadbalance daemon and assign IRQs manually
\item disable netfilter completely by systemctl disable firewalld or at least disable connection tracking (iptables)
\end{itemize}
