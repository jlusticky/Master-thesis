%=========================================================================
% (c) 2014, 2015 Josef Lusticky

\chapter{Routing subsystem}
---
Forwarding information base.
Linux kernel supports multiple routing tables.
There are two routing tables by default (in case of non policy routing):
local FIB table
main FIB table

include/net/ip_fib.h

Regardless of how many there are routing tables, there is always one routing cache ("route -C").


Routes can be introduced to the main routing table in one of three ways:
by system administrator command ('ip route add' or 'route add' which is obsolete)
by routing daemons
as a result of ICMP Redirect

A routing table is implemented by struct fib_table.
There is one routing cache, regardless of how many routing tables there are.
The routing cache can be viewed by running 'ip route show cache' or 'route -C' (/proc/net/rt_cache).

The routing cache consists of rtable elements.


---
http://www.haifux.org/lectures/172/netLec.pdf


---

Linux routing can be manipulated using netlink interface.
This interface provides easier use than ioctl(2).
This interface provides traditional socket-based messages.
socket(AF_NETLINK, ...);
send(..);

Rtnetlink allows the kernel's routing tables to be read and altered.

netlink(3) - system call - netlink family NETLINK_ROUTE
libnetlink , http://www.infradead.org/~tgr/libnl/
