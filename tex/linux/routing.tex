%=========================================================================
% (c) 2011, 2012 Josef Lusticky

\section{Routing subsystem}\label{sec:linux-routing}
Routing takes place on L3, so it is entirely kernel's responsibility to forward packets to their destination.
For each packet, incoming or outgoing, a lookup in the routing subsystem is performed~\cite{linux-kernel-networking}.
The decision about whether a packet should be forwarded and which interface it should be sent on is done based on the result of
the lookup in the routing subsystem.
The routing subsystem is a component of the kernel's IP stack and is responsible for forwarding packets and maintaining
the forwarding information base (FIB)~\cite{linux-kernel-networking}.

Familiar routing daemons, such as Quagga or Bird, are entirely user-space applications.
They are not responsible for routing any packet.
Instead, these routing daemons manipulate the kernel's FIB to
contain the selected routes based on
the routing protocol and algorithm they use (OSPF link-state, BGP best path, etc.).
To use these protocols and algorithms,
these user-space daemons usually maintain routing tables of their own, which should not be confused with the
FIB used by the kernel~\cite{linux-kernel-networking}.
The kernel's FIB is manipulated from user-space
either by {\it{ioctl()}} or by modern {\it{Netlink sockets}}~\cite{networking-subsystem-configuration-interface}.

Advanced routing topics, such as multipath and multicast routing are not covered in this thesis.
Multipath routing provides ability to add more than one nexthop to a route~\cite{linux-kernel-networking}.
Otherwise only one nexthop can be specified for a destination.
Multicast routing provides ability to route packets destined to multicast addresses~\cite{linux-kernel-networking}.

The Linux kernel supports up to 255 routing tables that can be used for policy routing~\cite{linux-kernel-networking}.
%The compile-time configuration option {\it{CONFIG\_IP\_MULTIPLE\_TABLES}} enables support for multiple routing tables.
However, the use of multiple routing tables can make a router very complex and therefore
policy routing is beyond the scope of this thesis.
With no policy routing, there are two routing tables created by the kernel while booting - the {\it{local}} FIB table
and the {\it{main}} FIB table~\cite{linux-kernel-networking}.
The {\it{local}} table contains routing entries of local addresses.
Routing entries can be added to the local table only by the kernel.
Adding routing entries to the {\it{main}} table is done by a system
administrator or by routing daemons or as a result of an ICMP Redirect message~\cite{linux-kernel-networking}.

The routing entries of the kernel's FIB table are organised as a Trie structure~\cite{linux-kernel-networking}.
The routing lookup in the Linux kernel uses longest matching prefix lookup algorithm
called FIB TRIE (also known as LC-trie), which performs good for large routing tables~\cite{linux-kernel-networking}.
The routing lookup can consume much of CPU time, depending on the size of the FIB table.
It can also consume much of the memory as the algorithm is rather complex~\cite{linux-kernel-networking}.

A lookup is done by the {\it{fib\_lookup()}} function, defined in the {\it{include/net/ip\_fib.h}} file
of the kernel source code~\cite{kernel-source}.
When the {\it{fib\_lookup()}} function finds a proper entry in the FIB table,
it builds a {\it{fib\_result}} object, which consists of various routing parameters,
including the next hop associated with the outgoing interface~\cite{linux-kernel-networking}.
Listing~\ref{lst:linux-fib-lookup} shows implementation of {\it{fib\_lookup}} when
multiple routing tables configuration is disabled.
%compile-time option CONFIG\_IP\_MULTIPLE\_TABLES is disabled.

\bigskip
\begin{lstlisting}[caption={Implementation of the fib\_lookup() function},label={lst:linux-fib-lookup}]
int fib_lookup(struct net *net, const struct flowi4 *flp, struct fib_result *res)
{
  struct fib_table *table;

  table = fib_get_table(net, RT_TABLE_LOCAL);
  if (!fib_table_lookup(table, flp, res, FIB_LOOKUP_NOREF))
    return 0;

  table = fib_get_table(net, RT_TABLE_MAIN);
  if (!fib_table_lookup(table, flp, res, FIB_LOOKUP_NOREF))
    return 0;

  return -ENETUNREACH;
}
\end{lstlisting}

The {\it{flowi4}} object consists of fields that are important to the IPv4 routing lookup process, including the
destination address, source address, Type of Service (TOS), and more~\cite{linux-kernel-networking}.
In fact the {\it{flowi4}} object defines the key to the lookup in the routing tables.
For IPv6 there is a parallel object named {\it{flowi6}}.
Both are defined in the {\it{include/net/flow.h}} file of the kernel source code~\cite{kernel-source}.
The {\it{fib\_lookup()}} function first searches the {\it{local}} FIB table.
If the lookup fails, it performs a lookup in the {\it{main}} FIB table~\cite{linux-kernel-networking}.
If that fails as well, an error code representing network unreachable is returned.

After a lookup is successfully done, a destination entry
object is built and associated with the {\it{skb}}~\cite{linux-kernel-networking}.
The destination entry object is implemented by {\it{struct dst\_entry}}, defined in the {\it{include/net/dst.h}} file of
the kernel source code~\cite{kernel-source}.
The result of the lookup is referenced by the {\it{struct fib\_result *res}} pointer,
which indirectly references the created {\it{dst\_entry}} object.

The most important members of the {\it{dst\_entry}} structure are two callbacks named {\it{input}} and {\it{output}}.
These callbacks are assigned to be proper handlers according to the routing lookup result.
For incoming unicast packets destined to the local host, the {\it{input}} callback is set to
{\it{ip\_local\_deliver()}}, and for incoming packets that should be forwarded,
this input callback is set to {\it{ip\_forward()}}.
For a packet generated on the local machine and sent away,
the output callback is set to {\it{ip\_output()}}~\cite{linux-kernel-networking}.
Listing~\ref{lst:linux-dst} shows a part of the {\it{struct dst\_entry}} definition.
\bigskip
\begin{lstlisting}[caption={Destination callback members of struct dst\_entry},label={lst:linux-dst}]
struct dst_entry {
  . . .
  int (*input)(struct sk_buff *);
  int (*output)(struct sk_buff *);
  . . .
}
\end{lstlisting}

A reference to the {\it{dst\_entry}}, which was created as a result of the routing decision,
is assigned to the {\it{skb$\rightarrow$\_skb\_refdst}} member.
The {\it{ip\_rcv\_finish()}} function further calls the {\it{input}} callback
which passes the {\it{skb}} either to the local host processing or to forwarding, as described in section~\ref{sec:linux-ip}.
In case of forwarding, the {\it{output}} callback is further set to the {\it{ip\_output}} function by the
{\it{ip\_forward\_finish()}}, as described in section~\ref{sec:linux-ip},

In terms of performance, there is currently not much space for improvements as
each {\it{skb}} must be handled separately and must be passed to the described functions of the IP stack.
In kernels prior to 3.6, there was an IPv4 routing cache with a garbage collector~\cite{linux-kernel-networking}.
The IPv4 routing cache was removed in kernel 3.6 (released in July 2012),
as it proven to be inefficient and vulnerable to DoS attacks~\cite{linux-cache-removing}.

There used to be a feature called Fastroute that allowed device drivers to route incoming traffic during
interrupt context using a small cache.
The packets were forwarded to the outgoing interface
without having to pass through the higher layer (IP)~\cite{linux-kernel-networking}.
However, this feature is not compatible with other important features, such as Netfilter firewall or advanced routing,
for the simple reason that this low-level forwarding would bypass them.
Starting with the 2.6.8 kernel, Fastroute is no longer supported and
its implementation was removed from the Linux kernel~\cite{linux-kernel-networking}.

As discussed above, the L3 packet processing and routing handle packets associated with their own {\it{skb}}.
However, the lower-layer part of the networking code can provide significant improvements
before the {\it{skb}} enters the {\it{ip\_rcv()}} function.
The improvements depend on how this code handles the ingress frames.
