%=========================================================================
% (c) 2014, 2015 Josef Lusticky

\chapter{Introduction}
The growth of Ethernet from 10 Mbit/s to 10 Gbit/s has already surpassed
the growth of microprocessor performance.
The 40~Gigabit Ethernet makes the performance gap even larger, but
it is still the original Ethernet underneath - an old technology
with a lot of compatibility issues for high-speed packet processing.
The recent 40 and 100~Gigabit Ethernet standard opens doors to
high-speed networking, but it requires other parts of the network to scale within.

The GNU/Linux operating system is used in a wide range of computers interconnected with high-speed Ethernet.
An important task of the Linux network stack is to forward traffic.
This is relevant especially when discussing core routers, which operate in the Internet backbone.
Forwarding occurs on Layer~3 of the ISO/OSI network model.
The performance of a software-based solution that uses GNU/Linux, cannot compete
with commercial products that can count on the help of specialised hardware.
However, various stack bypass solutions have shown, that the Linux kernel is not using
the CPU optimally.

The purpose of the thesis is to provide a comprehensive performance analysis of the Linux kernel
in IP packet forwarding.
The 40~Gigabit Ethernet protocol is able to transmit up to 59~million frames per second
and 4.6~GB of L2 data per second.
Such speed can easily burden the CPU with a large amount of TCP/IP protocol processing required.

Apart from the 40~Gigabit Ethernet protocol itself,
the packet processing in the Linux kernel is described in the thesis.
Since the emerge of 100~Mbps Ethernet, the Linux kernel
engineers have been optimising the network stack towards high-speed packet processing.
Hardware vendors have made various optimisations,
which help the operating system to lower the amount of processing required.

In the thesis, a high-end server with a 40~GbE network interface card
and 2 Intel Xeon CPUs was setup to measure the routing performance of the Linux kernel.
The measurements presented in this thesis demonstrate performance influences of various system settings
such as scaling mechanisms, Reverse path filtering or SELinux.
The measurements presented in this thesis also include comparison of IPv6 processing performance
against IPv4 processing performance.
Additionally, the thesis presents the Linux routing performance with imported routes from the Internet BGP protocol.
At the time of writing, there are approx.~538~000 routes announced in the public BGP,
which leads to expensive software lookups in the Forwarding Information Base of the Linux kernel.

Spirent hardware packet generator was used to generate the traffic and to collect the results.
Unfortunately, the measurements must have been configured manually, since the provided Spirent does not
contain licenses to automate the testing scenarios.

Measuring performance of the software IP routing using GNU/Linux-based operating system on 40~Gigabit Ethernet
can reveal bottlenecks that need to be eliminated
on the way to a full-speed 40~Gigabit TCP/IP processing.
If the system processes packets on Layer~3 fast enough,
next step is to optimise TCP and higher Layers of the network stack.
