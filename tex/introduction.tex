%=========================================================================
% (c) 2014, 2015 Josef Lusticky

\chapter{Introduction}
A fundamental obstacle to improving network performance is that servers were designed
for computing rather than input and output (I/O).
The Internet revolution has drastically changed server requirements,
and I/O is becoming a major bottleneck in delivering high-speed computing.
Not only the global Internet traffic is increasing, but also middle and small enterprises
need bandwidth and latency comparable to what ISPs used a few years ago.

The recent 40 and 100~Gigabit Ethernet standard provides solution
to handle the growing traffic within enterprise campus networks.
Thanks to its cheap price, compatibility and wide-spread use,
Ethernet has become ubiquitous data link protocol.
However, the growth of Ethernet from 10 Mbit/s to 10 Gbit/s has already surpassed
the growth of microprocessor performance. %~\cite{10gea-toe}.
The 40~Gigabit Ethernet makes the performance gap even larger, but
it is still the original Ethernet underneath - an old technology
with a lot of compatibility issues for high-speed networking.

Another bottleneck is the TCP/IP stack being processed at a rate less than the network speed.
The processing of TCP/IP over Ethernet is traditionally accomplished by software running on the CPUs of the server.
As network connections scale beyond Gigabit Ethernet speeds,
the CPU becomes burdened with a large amount of TCP/IP protocol processing required.
Reassembling out-of-order packets, resource-intensive memory copies, and interrupts put a tremendous load on the host's CPU.

The GNU/Linux operating system is used in a wide range of high-speed servers.
An important task of the Linux network stack is to forward traffic.
This is relevant especially when discussing core routers, which operate in the Internet backbone.
Forwarding occurs on Layer~3 of the ISO/OSI network model.
The performance of a pure software-based solution that uses Linux, cannot compete
with commercial products that can count on the help of specialised hardware.
This is a consequence of the speed difference between dedicated hardware and general-purpose CPUs.
However, high performance packet forwarding opens doors to handling multiple-gigabit Ethernet speeds -
if the system processes packets on Layer~3 fast enough,
next step is to optimise TCP and higher Layers of the network stack.

Measuring performance of the software IP routing using GNU/Linux-based operating system on 40~Gigabit Ethernet
can reveal bottlenecks that need to be eliminated
on the way to a full-speed 40~Gigabit TCP/IP processing.
