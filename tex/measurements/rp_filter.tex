%=========================================================================
% (c) 2014, 2015 Josef Lusticky

\subsection{Reverse path filter}

Enable rp\_filter.
\begin{lstlisting}[language=TeX]
echo 1 | tee /proc/sys/net/ipv4/conf/*/rp_filter
\end{lstlisting}

\begin{tabular}{ | l | l | l | l | }
\hline
Frame size & \% of link & bandwidth & frame rate \\
\hline
AMS-IX & 55.54\% & 22.21~Gb/s & 3~650~000 \\
\hline
\end{tabular}


perf top -C 11
\begin{lstlisting}
  39.88%  [kernel]  [k] fib_table_lookup
   8.35%  [kernel]  [k] check_leaf.isra.7
   6.43%  [kernel]  [k] _raw_spin_lock
   2.94%  [kernel]  [k] mlx4_en_xmit
   2.40%  [kernel]  [k] memcpy
   2.32%  [kernel]  [k] __netif_receive_skb_core
   2.18%  [kernel]  [k] mlx4_en_process_rx_cq
   2.14%  [kernel]  [k] fib_validate_source
   1.40%  [kernel]  [k] ip_route_input_noref
\end{lstlisting}
The fib\_table\_lookup is performed twice for each packet
and it is therefore taking much more of the CPU time.
There is also new fib\_validate\_source function, which is perfoming calling
the actual fib\_table\_lookup.
