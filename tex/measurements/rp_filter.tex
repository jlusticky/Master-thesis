%=========================================================================
% (c) 2014, 2015 Josef Lusticky

%FINAL
\subsection{Reverse path filter}
The measurement features Reverse path filter enabled, which is described in section~\ref{sec:analysis-settings}.
The rp\_filter can be enabled via {\it{procfs}}.
\begin{lstlisting}[language=TeX]
echo 1 | tee /proc/sys/net/ipv4/conf/*/rp_filter
\end{lstlisting}

\begin{tabular}{ | l | l | l | l | }
\hline
Frame size & \% of link & bandwidth & frame rate \\
\hline
AMS-IX & 55.54\% & 22.21~Gb/s & 3~650~000 \\
\hline
\end{tabular}
\\
\\
The rp\_filter introduces a significant performance drop of about 37\%.
The following listing shows the ouput of the {\it{perf}} utility.
\begin{lstlisting}
perf top -C 10
  39.88%  [kernel]  [k] fib_table_lookup
   8.35%  [kernel]  [k] check_leaf.isra.7
   6.43%  [kernel]  [k] _raw_spin_lock
   2.94%  [kernel]  [k] mlx4_en_xmit
   2.40%  [kernel]  [k] memcpy
   2.32%  [kernel]  [k] __netif_receive_skb_core
   2.18%  [kernel]  [k] mlx4_en_process_rx_cq
   2.14%  [kernel]  [k] fib_validate_source
   1.40%  [kernel]  [k] ip_route_input_noref
\end{lstlisting}
The {\it{fib\_table\_lookup}} is performed twice for each packet
and it is therefore taking much more of the CPU time.
There is also new {\it{fib\_validate\_source}} function, which calls
the actual {\it{fib\_table\_lookup}}.

In bidirectional routing with Reverse path filter enbaled,
it may be worth changing the default RSS hash key to a symetric one.
The RSS hash key is a taken as input by the Toeplitz hash function, as described in section~\ref{sec:linux-scaling}.
A symetric RSS key would lead to processing both directions of the same flow on the same CPU,
therefore the result of the {\it{fib\_table\_lookup}} could be fetched from the CPU's cache~\cite{symetric-rss}.
