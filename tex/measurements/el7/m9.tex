%=========================================================================
% (c) 2014, 2015 Josef Lusticky

% FINAL
\subsection{Measurement 9 - 32 IPv6 flows with manual IRQ affinity mappings}
The measurement can be directly compared to Measurement 7.
%Since the previous measurement reported 10\% performance drop against IPv4, it is expected....
\\
\\
\begin{tabular}{ | l | l | l | l | }
\hline
Frame size & \% of link & bandwidth & frame rate \\
\hline
78     &  5.88\% &  2.35~Gb/s & 3~000~000 \\
594    & 36.84\% & 14.74~Gb/s & 3~550~000 \\
1518   & 98.50\% & 39.40~Gb/s & 3~202~210 \\
AMS-IX & 54.77\% & 21.91~Gb/s & 3~600~000 \\
\hline
\end{tabular}
\\
\\
The measurement featuring 1518~octet frames was configured to use 98.50\% of the link bandwidth.
The results show a significant performance drop when routing multiple IPv6 flows.
\\
This performance drop can be investigated by changing the number of flows to 8
and observing the interrupt distribution.
The following listing shows the partial output of the /proc/interrupts file when routing 8 flows.
\begin{lstlisting}
      ... CPU12 CPU13 CPU14 CPU15 CPU16 CPU17 CPU18 ...
 178: ...     0     0     0     0     0     0     0 ... enp129s0d1-0
 179: ...     0     0     0     0     0     0     0 ... enp129s0d1-1
 180: ...     0     0     0     0     0     0     0 ... enp129s0d1-2
 181: ...     0 32513     0     0     0     0     0 ... enp129s0d1-3
 182: ...     0     0     0     0     0     0     0 ... enp129s0d1-4
 183: ...     0     0     0     0     0     0     0 ... enp129s0d1-5
 184: ...     0     0     0     0     0     0     0 ... enp129s0d1-6
 185: ...     0     0     0     0     0 33543     0 ... enp129s0d1-7
 186: ...     0     0     0     0     0     0     0 ... enp129s0-0
 187: ...     0     0     0     0     0     0     0 ... enp129s0-1
 188: ...     0     0     0     0     0     0     0 ... enp129s0-2
 189: ...     0 19384     0     0     0     0     0 ... enp129s0-3
 190: ...     0     0     0     0     0     0     0 ... enp129s0-4
 191: ...     0     0     0     0     0     0     0 ... enp129s0-5
 192: ...     0     0     0     0     0     0     0 ... enp129s0-6
 193: ...     0     0     0     0     0 19576     0 ... enp129s0-7
\end{lstlisting}
In contrast to IPv4, the RSS does not distribute IPv6 packets uniformly.
As a result of this, the routing performance of IPv6 protocol is lower than in case of IPv4.
The traffic is also not distributed uniformly when routing 32 flows,
however, this issue cannot be detected from reading the /proc/interrupts file due to
interrupt mitigation mechanism used by NAPI, as described in section~\ref{sec:linux-ingress}.
