%=========================================================================
% (c) 2014, 2015 Josef Lusticky

% FINAL
\subsection{Measurement 8 - single IPv6 flow}
The measurement serves as a comparison between IPv4 and IPv6 processing performance.
The measurement can be directly compared to Measurement~3.
\\
\\
\begin{tabular}{ | l | l | l | l |}
\hline
Frame size & \% of link & bandwidth & frame rate \\
\hline
78     &  1.76\% &  0.71~Gb/s & 900~000 \\
594    & 11.05\% &  4.42~Gb/s & 900~000 \\
1518   & 27.68\% & 11.07~Gb/s & 900~000 \\
AMS-IX & 13.69\% &  5.48~Gb/s & 900~000 \\
\hline
\end{tabular}
\\
\\
The IPv6 processing performance is 10\% lower than the IPv4 processing performance
when forwarding traffic on a single core.
\\
The following listing shows the output of the {\it{perf}} utility:
\begin{lstlisting}
perf top -C 10
  11.75%  [kernel]  [k] ip6t_do_table
  10.37%  [kernel]  [k] _raw_spin_lock
   8.43%  [kernel]  [k] fib6_lookup
   4.93%  [kernel]  [k] ip6_forward
   3.66%  [kernel]  [k] fib6_get_table
   3.48%  [kernel]  [k] ip6_rcv_finish
   3.40%  [kernel]  [k] build_skb
   3.26%  [kernel]  [k] __netif_receive_skb_core
   3.01%  [kernel]  [k] mlx4_en_complete_rx_desc
   3.00%  [kernel]  [k] _raw_read_unlock_bh
   2.65%  [kernel]  [k] mlx4_en_process_rx_cq
   2.62%  [kernel]  [k] dst_release
\end{lstlisting}
As in the case of IPv4, the CPU spends most of the time on the actual lookup and locking.
