%=========================================================================
% (c) 2014, 2015 Josef Lusticky

% FINAL
\subsection{Measurement 6 - 32 IPv4 flows with irqbalance daemon}
The measurement includes the {\it{irqbalance}} daemon enabled.
The {\it{irqbalance}} daemon is responsible for dynamically assigning the interrupts to CPUs
using the files found under /proc/irq/{\it{NUMBER}}/smp\_affinity,
as described in section~\ref{sec:analysis-settings}.
\\
\\
\begin{tabular}{ | l | l | l | l | }
\hline
Frame size & \% of link & bandwidth & frame rate \\
\hline
64     &  8.99\% &  3.60~Gb/s & 5~350~000 \\
594    & 68.15\% & 27.26~Gb/s & 5~550~000 \\
1518   & 98.50\% & 39.40~Gb/s & 3~202~210 \\
AMS-IX & 88.25\% & 35.30~Gb/s & 5~800~000 \\
\hline
\end{tabular}
\\
\\
The scaling mechanisms of the Linux kernel take advantage of interrupt assignment
done by the {\it{irqbalance}} daemon.
The server is able to route almost 36~Gbps of the simulated AMS-IX internet traffic.
The measurement further confirms that the scaling mechanisms are sensitive to the frame size.
The measurement featuring 1518~octet frames was configured to use 98.5\% of the link bandwidth.
\\
The following listing shows that the {\it{irqbalance}} daemon assigned IRQs to CPUs 11-18, 30 and 33-39.
The CPUs 12-19 are serving RX interrupts, while the CPUs 30-39 are serving TX interrupts
({\it{en29d1}} represents the receiving interface).

\newpage

\begin{landscape}
\vspace*{\fill}
\begin{lstlisting}
     CPU12 CPU13 CPU14 CPU15 CPU16 CPU17 CPU18 CPU19  CPU30  CPU33  CPU34  CPU35  CPU36  CPU37  CPU38  CPU39
178:     0     0     0     0     0     0     0     0      0 292448      0      0      0      0      0      0 en29-0
179:     0     0     0     0     0     0     0     0      0      0 292978      0      0      0      0      0 en29-1
180:     0     0     0     0     0     0     0     0      0      0      0 292698      0      0      0      0 en29-2
181:     0     0     0     0     0     0     0     0      0      0      0      0 286435      0      0      0 en29-3
182:     0     0     0     0     0     0     0     0      0      0      0      0      0 282449      0      0 en29-4
183:     0     0     0     0     0     0     0     0      0      0      0      0      0      0 288839      0 en29-5
184:     0     0     0     0     0     0     0     0      0      0      0      0      0      0      0 327901 en29-6
185:     0     0     0     0     0     0     0     0 325935      0      0      0      0      0      0      0 en29-7
186: 53145     0     0     0     0     0     0     0      0      0      0      0      0      0      0      0 en29d1-0
187:     0 53090     0     0     0     0     0     0      0      0      0      0      0      0      0      0 en29d1-1
188:     0     0 39978     0     0     0     0     0      0      0      0      0      0      0      0      0 en29d1-2
189:     0     0     0 40484     0     0     0     0      0      0      0      0      0      0      0      0 en29d1-3
190:     0     0     0     0 40072     0     0     0      0      0      0      0      0      0      0      0 en29d1-4
191:     0     0     0     0     0 39982     0     0      0      0      0      0      0      0      0      0 en29d1-5
192:     0     0     0     0     0     0 40488     0      0      0      0      0      0      0      0      0 en29d1-6
193:     0     0     0     0     0     0     0 43262      0      0      0      0      0      0      0      0 en29d1-6
\end{lstlisting}
\vspace*{\fill}
\end{landscape}

\noindent
\\
The listing bellow shows the cache use.
\begin{lstlisting}[language=TeX]
 L3MISS: L3 cache misses 
 L2MISS: L2 cache misses (including other core's L2 cache *hits*) 
 L3HIT : L3 cache hit ratio (0.00-1.00)
 L2HIT : L2 cache hit ratio (0.00-1.00)
 L3CLK : ratio of CPU cycles lost due to L3 cache misses (0.00-1.00), in some cases could be >1.0 due to a higher memory latency
 L2CLK : ratio of CPU cycles lost due to missing L2 cache but still hitting L3 cache (0.00-1.00)


 Core (SKT) | L3MISS | L2MISS | L3HIT | L2HIT | L3CLK | L2CLK |  L3OCC

   0    0     1076       11 K    0.91    0.17    0.01    0.01      120
   1    0      574     4139      0.86    0.12    0.10    0.14       40
   2    0      241     1421      0.83    0.18    0.12    0.12        0
   3    0      658       11 K    0.94    0.09    0.07    0.24        0
   4    0       19      580      0.97    0.14    0.02    0.12        0
   5    0       78      363      0.79    0.25    0.03    0.03        0
   6    0       65      710      0.91    0.15    0.04    0.10       40
   7    0       19      544      0.97    0.12    0.02    0.10       40
   8    0       13      648      0.98    0.11    0.01    0.08        0
   9    0       32      582      0.95    0.11    0.02    0.08        0
  10    1     1007     3598      0.72    0.06    0.72    0.37       40
  11    1     4452     3802 K    1.00    0.45    0.00    0.10       40
  12    1     4932     3770 K    1.00    0.42    0.00    0.10      240
  13    1     6273     4131 K    1.00    0.51    0.00    0.09       40
  14    1     5086     4211 K    1.00    0.52    0.00    0.09       80
  15    1     4762     4211 K    1.00    0.50    0.00    0.10       80
  16    1     4453     4111 K    1.00    0.54    0.00    0.09       80
  17    1     4680     4124 K    1.00    0.56    0.00    0.09      160
  18    1     4737     4275 K    1.00    0.48    0.00    0.10       80
  19    1      105      573      0.82    0.18    0.18    0.17        0
  20    0      170      979      0.83    0.06    0.07    0.10       40
  21    0       16      692      0.98    0.10    0.01    0.12        0
  22    0       12      570      0.98    0.12    0.01    0.13        0
  23    0       15      839      0.98    0.10    0.01    0.13        0
  24    0       13      532      0.98    0.16    0.02    0.14        0
  25    0       22      441      0.95    0.21    0.01    0.06        0
  26    0       14      540      0.97    0.14    0.02    0.13        0
  27    0        9      516      0.98    0.15    0.01    0.13        0
  28    0      268     2659      0.90    0.09    0.10    0.18       40
  29    0      239      916      0.74    0.15    0.10    0.06        0
  30    1     1238     2816 K    1.00    0.60    0.00    0.27      680
  31    1       54      560      0.90    0.25    0.09    0.19       80
  32    1     3416     9666      0.65    0.20    0.72    0.27       40
  33    1       14 K   3887 K    1.00    0.64    0.00    0.27      640
  34    1       16 K   3969 K    1.00    0.63    0.00    0.27      960
  35    1       15 K   3881 K    1.00    0.64    0.00    0.27      960
  36    1       14 K   3878 K    1.00    0.63    0.00    0.27      880
  37    1       14 K   3920 K    1.00    0.63    0.00    0.27      440
  38    1       17 K   4014 K    1.00    0.63    0.00    0.27      600
  39    1     2106     2853 K    1.00    0.60    0.00    0.28     1240
------------------------------------------------------------------------
 SKT    0     3553       41 K    0.91    0.13    0.01    0.03      320
 SKT    1      141 K     61 M    1.00    0.57    0.00    0.14     7360
------------------------------------------------------------------------
 TOTAL  *      144 K     61 M    1.00    0.57    0.00    0.14      N/A 
\end{lstlisting}
The measurement featuring 1518~octet frames is the first measurements saturating the 40~Gbps Ethernet connection.
Intel PCM can be used to monitor the PCI-Express utilisation:
\begin{lstlisting}
Skt | PCIe Rd (B) | PCIe Wr (B)
 0         5270 K          86 K
 1           11 G        5422 M
--------------------------------
 *           11 G        5422 M
\end{lstlisting}
The PCI-Express link could be saturated when forwarding bidirectional traffic
- the PCI-Express 3.0 x8 throughput is 7~876.8~MB/s as calculated in section~\ref{sec:40gbe-throughput}.
Note, that there seems to be a bug in Intel PCM related to displaying
the PCIe Read bandwidth - it always shows double the expected value (reading 11~Gigabytes does not make sense).

%Irqbalance moves the interrupts between CPUs.
%Heavily depends on the dynamic IRQ mappings.
%Executing the hardware interrupt handler on new CPU that has not served the interrupt yet causes cache miss.
