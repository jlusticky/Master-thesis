%=========================================================================
% (c) 2014, 2015 Josef Lusticky

% FINAL
\subsection{Measurement 4 - 32 independent IPv4 flows}
This measurement can be compared with Measurement~2, except that the {\it{irqbalance}} daemon is disabled.
Therefore, the IRQ mapping is left untouched in its default state, as shown by Measurement~3.
\begin{tabular}{ | l | l | l | l | }
\hline
Frame size & \% of link & bandwidth & frame rate \\
\hline
64     &  1.26\% &  0.50~Gb/s & 750~000 \\
594    & 12.28\% &  4.91~Gb/s & 800~000 \\
1518   & 24.61\% &  9.84~Gb/s & 800~000 \\
AMS-IX & 15.22\% &  6.09~Gb/s & 800~000 \\
\hline
\end{tabular}
When forwarding IP packets from multiple IPv4 flows on a single CPU,
the routing performance of the Linux kernel drops by 20\% against forwarding a single IPv4 flow.
%Since the destination IP address is the same as in the previous case,
%this is the cost of fetching the packets from different ingress queues.
Each receive queue triggers roughly the same number of interrupts as in the previous measurement,
but overall the NIC triggers much more interrupts.
The kernel spends more time on running the interrupt service routine code.

Interrupt mapping:
\begin{lstlisting}
      ... CPU8  CPU9   CPU10 CPU11 CPU12  ...
178:  ...    0     0  474701     0     0  ...  enp129s0-0
179:  ...    0     0       0     0     0  ...  enp129s0-1
180:  ...    0     0       0     0     0  ...  enp129s0-2
181:  ...    0     0       0     0     0  ...  enp129s0-3
182:  ...    0     0       0     0     0  ...  enp129s0-4
183:  ...    0     0       0     0     0  ...  enp129s0-5
184:  ...    0     0       0     0     0  ...  enp129s0-6
185:  ...    0     0       0     0     0  ...  enp129s0-7

186:  ...    0     0  317322     0     0  ...  enp129s0d1-0
187:  ...    0     0  317648     0     0  ...  enp129s0d1-1
188:  ...    0     0  317231     0     0  ...  enp129s0d1-2
189:  ...    0     0  317384     0     0  ...  enp129s0d1-3
190:  ...    0     0  317114     0     0  ...  enp129s0d1-4
191:  ...    0     0  317291     0     0  ...  enp129s0d1-5
192:  ...    0     0  317190     0     0  ...  enp129s0d1-6
193:  ...    0     0  317964     0     0  ...  enp129s0d1-7
\end{lstlisting}
The packets are evenly distributed among all hardware queues.
However, the interrupts are not distributed among CPUs.

Processor usage:
\begin{lstlisting}
  12.07%  [kernel]  [k] _raw_spin_lock
   8.68%  [kernel]  [k] fib_table_lookup
   5.01%  [kernel]  [k] mlx4_en_xmit
   4.63%  [kernel]  [k] mlx4_en_process_rx_cq
   3.64%  [kernel]  [k] __netif_receive_skb_core
   3.49%  [kernel]  [k] memcpy
   3.08%  [kernel]  [k] irq_entries_start
   2.68%  [kernel]  [k] mlx4_eq_int
   2.33%  [kernel]  [k] mlx4_en_poll_tx_cq
   2.24%  [kernel]  [k] ip_route_input_noref
\end{lstlisting}
The kernel spends most of the time on locking and FIB table lookup. 
